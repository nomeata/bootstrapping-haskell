\sect{User guide to {\tt pphs}}

The program {\tt pphs} typesets programs in the Haskell programming
language for use with the \LaTeX\ intensional text formatting
and typesetting system.  It takes
as input a file containing a Haskell program and produces \LaTeX\
code to {\tt stdout}.  There are various different features of this
process.

\subsect{Left indentation}

It is in the nature of Haskell programs that indentation is heavily used.  As the
indentation is vital to the parsing of the program, any attempt at typesetting
Haskell code must replicate this indentation.  Take, for example, the following piece of code.
\begin{quote}
% From cvh/Public/GBC/Source/Gm7.hs
\begin{verbatim}
eval :: GmState -> [GmState]
eval state = state: restStates
             where
             restStates | gmFinal state = []
                        | otherwise = eval nextState
             nextState  = doAdmin (step state)
\end{verbatim}

\end{quote}
Note how the third, fifth and sixth lines start at different levels of indentation.
The {\tt pphs} program produces the correct \LaTeX\ code to align these under the
correct position in the preceding lines once typeset.  It also selects the correct
line to line up under.  Note how the sixth line does not line up
under its predecessor, but under the fourth line.
The code necessary to typeset this is produced, preserving the parsing
order.  Once typeset, it will look like this:
\begin{quote}
\begin{tabbing}
{\iden eval\/}\xspa{1}$::$\xspa{1}{\iden GmState\/}\xspa{1}$\rightarrow$\xspa{1}$[${\iden GmState\/}$]$\\
{\iden eval\/}\xspa{1}{\iden state\/}\xspa{1}$=$\xspa{1}{\iden state\/}$:$\xspa{1}{\iden restStates\/}\\
\skipover{{\iden eval\/}\xspa{1}{\iden state\/}\xspa{1}$=$\xspa{1}}{\keyword where\/}\\
\skipover{{\iden eval\/}\xspa{1}{\iden state\/}\xspa{1}$=$\xspa{1}}{\iden restStates\/}\xspa{1}$|$\xspa{1}{\iden gmFinal\/}\xspa{1}{\iden state\/}\xspa{1}$=$\xspa{1}$[]$\\
\skipover{{\iden eval\/}\xspa{1}{\iden state\/}\xspa{1}$=$\xspa{1}{\iden restStates\/}\xspa{1}}$|$\xspa{1}{\iden otherwise\/}\xspa{1}$=$\xspa{1}{\iden eval\/}\xspa{1}{\iden nextState\/}\\
\skipover{{\iden eval\/}\xspa{1}{\iden state\/}\xspa{1}$=$\xspa{1}}{\iden nextState\/}\xspa{2}$=$\xspa{1}{\iden doAdmin\/}\xspa{1}$(${\iden step\/}\xspa{1}{\iden state\/}$)$
\end{tabbing}

\end{quote}
Note that this
example of possible input had no `extra' typesetting commands.

A line of Haskell code may be indented beyond the end of its predecessor.
Here, {\tt pphs} aligns it with whichever line it is lined up underneath in the
original file, or, if longer than any preceding line, inserts space to correspond
to that in the input file.

\subsect{Internal alignment}

Another form of alignment used in Haskell is {\em internal alignment}.  This is where
there is vertical alignment of columns other than at the left-hand edge of the
Haskell code.  This is typically characterised with a column of the same character
appearing in the program code, and it is this case, along with a
special case, that {\tt pphs} recognises for internal alignment having occurred.
\begin{quote}
% From Haskell report PreludeComlex.hs
\begin{verbatim}
instance  (RealFloat a) => Num (Complex a)  where
    (x:+y) + (x':+y')   =  (x+x') :+ (y+y')
    (x:+y) - (x':+y')   =  (x-x') :+ (y-y')
    (x:+y) * (x':+y')   =  (x*x'-y*y') :+ (x*y'+y*x')
    negate (x:+y)       =  negate x :+ negate y
    abs z               =  magnitude z :+ 0
    signum 0            =  0
    signum z@(x:+y)     =  x/r :+ y/r  where r = magnitude z
    fromInteger n       =  fromInteger n :+ 0
\end{verbatim}
\end{quote}
In this example, see how the {\tt =} signs line up, one below the other.  This makes
the program more readable, although it does not affect the parsing of the program.
As the purpose of {\tt pphs} is to make Haskell programs even more readable, it
retains this alignment.  This example would be typeset to produce the following:
\begin{quote}
\begin{tabbing}
\begin{tabular}{@{}l@{\xspa1}c@{}l}
{\keyword instance\/}\xspa{2}$(${\iden RealFloat\/}\xspa{1}{\iden a\/}$)$\xspa{1} & $\Rightarrow$ & \xspa{1}{\iden Num\/}\xspa{1}$(${\iden Complex\/}\xspa{1}{\iden a\/}$)$\xspa{2}{\keyword where\/}\\
\skipover{{\keyword inst\/}}$(${\iden x\/}{:}{+}{\iden y\/}$)$\xspa{1}$+$\xspa{1}$(${\iden x\/}$'${:}{+}{\iden y\/}$')$\xspa{3} & $=$ & \xspa{2}$(${\iden x\/}$+${\iden x\/}$')$\xspa{1}{:}{+}\xspa{1}$(${\iden y\/}$+${\iden y\/}$')$\\
\skipover{{\keyword inst\/}}$(${\iden x\/}{:}{+}{\iden y\/}$)$\xspa{1}$-$\xspa{1}$(${\iden x\/}$'${:}{+}{\iden y\/}$')$\xspa{3} & $=$ & \xspa{2}$(${\iden x\/}$-${\iden x\/}$')$\xspa{1}{:}{+}\xspa{1}$(${\iden y\/}$-${\iden y\/}$')$\\
\skipover{{\keyword inst\/}}$(${\iden x\/}{:}{+}{\iden y\/}$)$\xspa{1}$\times $\xspa{1}$(${\iden x\/}$'${:}{+}{\iden y\/}$')$\xspa{3} & $=$ & \xspa{2}$(${\iden x\/}$\times ${\iden x\/}$'-${\iden y\/}$\times ${\iden y\/}$')$\xspa{1}{:}{+}\xspa{1}$(${\iden x\/}$\times ${\iden y\/}$'+${\iden y\/}$\times ${\iden x\/}$')$\\
\skipover{{\keyword inst\/}}{\iden negate\/}\xspa{1}$(${\iden x\/}{:}{+}{\iden y\/}$)$\xspa{7} & $=$ & \xspa{2}{\iden negate\/}\xspa{1}{\iden x\/}\xspa{1}{:}{+}\xspa{1}{\iden negate\/}\xspa{1}{\iden y\/}\\
\skipover{{\keyword inst\/}}{\iden abs\/}\xspa{1}{\iden z\/}\xspa{15} & $=$ & \xspa{2}{\iden magnitude\/}\xspa{1}{\iden z\/}\xspa{1}{:}{+}\xspa{1}{\numb 0\/}\\
\skipover{{\keyword inst\/}}{\iden signum\/}\xspa{1}{\numb 0\/}\xspa{12} & $=$ & \xspa{2}{\numb 0\/}\\
\skipover{{\keyword inst\/}}{\iden signum\/}\xspa{1}{\iden z@\/}$(${\iden x\/}{:}{+}{\iden y\/}$)$\xspa{5} & $=$ & \xspa{2}{\iden x\/}$/${\iden r\/}\xspa{1}{:}{+}\xspa{1}{\iden y\/}$/${\iden r\/}\xspa{2}{\keyword where\/}\xspa{1}{\iden r\/}\xspa{1}$=$\xspa{1}{\iden magnitude\/}\xspa{1}{\iden z\/}\\
\skipover{{\keyword inst\/}}{\iden fromInteger\/}\xspa{1}{\iden n\/}\xspa{7} & $=$ & \xspa{2}{\iden fromInteger\/}\xspa{1}{\iden n\/}\xspa{1}{:}{+}\xspa{1}{\numb 0\/}
\end{tabular}
\end{tabbing}

\end{quote}
The special case for internal alignment is a $=$ aligned under a $::$.
This will cause the same effect as would have happened if they were the same
character.

\subsect{Token highlighting}

To increase the readability of Haskell programs, {\tt pphs} allows various tokens
to be highlighted.  By using different typefaces for some pieces of code, this
distinguishes them from the rest.  The user can specify the details of
the highlighting, but the default settings are {\bf bold} for
keywords, {\it italics} for identifiers and {\rm roman} for everything else.
Strings, comments and numbers are also highlightable.

Note that in the previous example, the keywords {\bf instance} and {\bf where}
are highlighted in bold, whereas the various identifiers are in italics.

\subsect{Mathematical symbols}

Rather than simply replicate the ASCII approximations of mathematical symbols
used in Haskell, {\tt pphs}
substitutes the proper symbols in the output.  These are shown below.
\begin{center}
\begin{tabular}[t]{|c|c|} \hline
{\em Haskell\/} & {\em Math\/} \\ \hline
{\tt *} & $\times$ \\
{\tt ++} & {\hbox{$+\mkern-7.5mu+$}} \\
{\tt :+} & {:}{+} \\
{\tt <=} & $\leq$ \\ \hline
\end{tabular} \hskip3mm \begin{tabular}[t]{|c|c|} \hline
{\em Haskell\/} & {\em Math\/} \\ \hline
{\tt >=} & $\geq$ \\
{\tt <-} & $\leftarrow$ \\
{\tt ->} & $\rightarrow$ \\
{\tt =>} & $\Rightarrow$ \\ \hline
\end{tabular}
\end{center}

\subsect{\LaTeX\ typesetting characters}

\LaTeX\ uses embedded typesetting commands, so {\tt pphs} has to ensure that if
any of the characters used by \LaTeX\ appear in the input Haskell code, the correct
\LaTeX\ code is outputted to typeset them, rather than have the characters interfere
with the typesetting process.  The characters used by \LaTeX\ for typesetting are:
\begin{quote}
\(\#\ \$\ \%\ \&\ \char'176\ \_\ \char'136\ \hbox{$\setminus$}\ \hbox{$\cal \char'146\ \char'147$}\)
\end{quote}
The user of {\tt pphs} need not worry about using any of these characters in Haskell
programs, as this will be dealt with by {\tt pphs} before \LaTeX\ gets to see the code.

\subsect{How to call it}

The program is called by typing {\tt pphs} followed by the name of
the file containing the Haskell program to be typeset.  If the
filename ends with a {\tt .hs} extension, this may be omitted,
unless another file exists with the same name but no extension.
When no extension is specified, the program will look for a
filename with no extension before looking for a file with the
{\tt .hs} extension.

For example, if the Haskell program was in a file called {\tt Haskell.hs},
the program would be called by
\begin{quote}
\tt pphs Haskell.hs
\end{quote}
As the filename ends with a {\tt .hs} extension, the extension may be omitted, provided
there is no file already existing called {\tt Haskell}.  If there is no such file
\begin{quote}
\tt pphs Haskell
\end{quote}
would produce the same effect as the original call.

As the program outputs to {\tt stdout}, the code produced may be
directed to a file by using a {\tt >} symbol after the call, followed by
the name of the file to contain the \LaTeX\ code produced by the
program.  Continuing the above example, if the output code is to be in
a file called {\tt Haskell.tex}, the call would now be
\begin{quote}
\tt pphs Haskell.hs > Haskell.tex
\end{quote}
It must be noted that if the file {\tt Haskell.tex} already exists, it must be
renamed or removed before making this call.

There are three options that can be specified in the program call.
If it is desired that double colon symbols should look like $:\,:$ rather than $::$,
use {\tt -w} in the call.  The length of the tab characters in the input file can
be specified with {\tt -t} followed by the length.  The default tablength is 8.
If identifiers with subscripts are wanted, eg {\iden ident$_1$\/}, then use {\tt -s}.
These are written in the Haskell file as {\tt ident\_1}.

If the length of the tabs are 4 and
the wide double colons are required, the example call above would become as follows.
\begin{quote}
\tt pphs -t4w Haskell.hs > Haskell.tex
\end{quote}

\subsect{What to do with the produced code}

Before including the \LaTeX\ code in the document, it is necessary
to include definitions of the \LaTeX\ commands used by {\tt pphs}.
This can be done simply by including the style file {\tt pphs.sty}
by adding {\tt pphs} to the option list of the documentstyle
command like thus:
\begin{quote}
\begin{verbatim}
\documentstyle[12pt,a4,pphs]{article}
\end{verbatim}
\end{quote}

Once this has been done, the file containing the \LaTeX\ code
of the Haskell program code can be included.  This is done
using the {\tt \char'134 input} command.  If the \LaTeX\
code is located in a file called {\tt Haskell.tex} then the
command is:
\begin{quote}
\begin{verbatim}
\input{Haskell}
\end{verbatim}
\end{quote}
This can be used in various \LaTeX\ environments such as {\tt quote},
{\tt figure} or {\tt table} to produce different effects. An example
of possible code is:
\begin{quote}
\begin{verbatim}
\begin{quote}
\input{Haskell}
\end{quote}
\end{verbatim}
\end{quote}
See Lamport, L.,  {\em \LaTeX : A Document Preparation System\/}
(Addison-Wesley, 1986) for more details.

\subsect{How to make adjustments to the output}

The {\tt pphs} program is flexible in that it allows user choice on some aspects
of the appearance of the final result.  User choice is allowed in two areas, typefaces
and qoute marks.

The default settings for typefaces are bold for keywords, italics for identifiers and
roman for everything else that is not in the math typeface.  However, keywords, identifiers,
strings, comments and numbers may be in whatever typeface the user chooses.
This is done using the {\tt \char'134 def} command to redefine the typeface commands
used by {\tt pphs}.  These are {\tt \char'134 keyword}, {\tt \char'134 iden},
{\tt \char'134 stri}, {\tt \char'134 com} and {\tt \char'134 numb} respectively.

For example, to put all comments into typewriter font, use
{\tt \char'134 def\char'134 com\char'173 \char'134 tt\char'175} in
the document.  The scope of the declaration will be from the point of introduction to
the end of the document.  To cancel a redefinition, use {\tt \char'134 def} to
redefine it back to what it was originally.  The different typefaces available in \LaTeX\ are
\begin{center}
\begin{tabular}{|c|l|} \hline
{\em code\/} & {\em meaning\/} \\ \hline
{\tt \char'134 bf} & {\bf Boldface} \\
{\tt \char'134 em} & {\em Emphatic\/} \\
{\tt \char'134 it} & {\it Italics\/} \\
{\tt \char'134 rm} & {\rm Roman} \\ \hline
\end{tabular} \hskip3mm \begin{tabular}{|c|l|} \hline
{\em code\/} & {\em meaning\/} \\ \hline
{\tt \char'134 sc} & {\sc Small Caps} \\
{\tt \char'134 sf} & {\sf Sans Serif} \\
{\tt \char'134 sl} & {\sl Slanted\/} \\
{\tt \char'134 tt} & {\tt Typewriter} \\ \hline
\end{tabular}
\end{center}
It should be noted that the emphatic typeface is just the same as italics, although
nesting emphatic sections will alternate between italics and roman.

Two types of quote mark are redefinable, forwards quotes and escape quotes.
The default for both of them is ' but if it is wished to redefine one or
both of them, use the {\tt \char'134 def} with either {\tt \char'134 forquo}
or {\tt \char'134 escquo}.  For example, to make escape quotes be
printed as {\sf '} use {\tt \char'134 def\char'134 escquo\char'173 \char'134 hbox\char'173 \char'134 sf '\char'175 \char'175} in the document.

\subsect{Altering the output}

As {\tt pphs} produces code which is subsequently run through \LaTeX , it is possible
to alter the code before it is run through \LaTeX .  This is useful for correcting
mistakes made by {\tt pphs}.  However, it is recommended that only those experienced
in \LaTeX\ try this.