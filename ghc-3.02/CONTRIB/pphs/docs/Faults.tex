\chapter{Things that don't work} \label{faults}

The {\tt pphs} program has some deficiencies that cause it to not always produce the
correct code.  These are detailed in this chapter.

\section{Internal alignment}

The program can deal only with simple internal alignment.  It cannot deal with a
situation where there is more than one column where internal alignment is occurring
on the same line.  This can occur when two sections of internal
alignment overlap by having lines in common or where one section is wholly within another.
When this happens, {\tt pphs} will only
line up one occurrence of internal alignment on each line.

Related is left alignment under a section of internal alignment.  Take this earlier example.
\begin{quote}
\begin{verbatim}
gcd       :: Int -> Int -> Int
gcd x y    = gcd' (abs x) (abs y)
             where gcd' x 0 = x
                   gcd' x y = gcd' y (x `rem` y)
\end{verbatim}


\end{quote}
This is how this code is typeset by {\tt pphs}:
\begin{quote}
\begin{tabbing}
\begin{tabular}{@{}l@{\xspa1}c@{}l}
{\iden gcd\/}\xspa{7} & $::$ & \xspa{1}{\iden Int\/}\xspa{1}$\rightarrow$\xspa{1}{\iden Int\/}\xspa{1}$\rightarrow$\xspa{1}{\iden Int\/}\\
{\iden gcd\/}\xspa{1}{\iden x\/}\xspa{1}{\iden y\/}\xspa{4} & $=$ & \xspa{1}{\iden gcd\/}$'$\xspa{1}$(${\iden abs\/}\xspa{1}{\iden x\/}$)$\xspa{1}$(${\iden abs\/}\xspa{1}{\iden y\/}$)$\\
\end{tabular}\\
\skipover{{\iden gcd\/}\xspa{1}{\iden x\/}\xspa{1}{\iden y\/}\xspa{4}$=$\xspa{1}}{\keyword where\/}\xspa{1}{\iden gcd\/}$'$\xspa{1}{\iden x\/}\xspa{1}{\numb 0\/}\xspa{1}$=$\xspa{1}{\iden x\/}\\
\skipover{{\iden gcd\/}\xspa{1}{\iden x\/}\xspa{1}{\iden y\/}\xspa{4}$=$\xspa{1}{\keyword where\/}\xspa{1}}{\iden gcd\/}$'$\xspa{1}{\iden x\/}\xspa{1}{\iden y\/}\xspa{1}$=$\xspa{1}{\iden gcd\/}$'$\xspa{1}{\iden y\/}\xspa{1}$(${\iden x\/}\xspa{1}$`${\iden rem\/}$`$\xspa{1}{\iden y\/}$)$
\end{tabbing}

\end{quote}
Notice how the {\bf where} on the third line doesn't line up under the {\it gcd\/}$'$ on
the second.  The reason for this
is the \LaTeX\ {\tt tabular} section does not respect any spaces that occur at the end
of the right hand edge of the left hand column such as those after
{\tt gcd x y} and instead moves the central column left
so it is only one space away from the longest piece of text in the left hand column,
in this case {\iden gcd\/}\xspa1 {\iden x\/}\xspa1 {\iden y\/}.
The left indentation of the lines under the internal alignment section does not take this
movement into account and so if a line is indented beyond the end of the text in the first
column of the last line of the internal alignment section then it may be incorrectly
positioned and therefore will not align with what it was aligned with in the original
program.  Should a piece of text in the left hand column be longer once typeset than what was
previously the longest, due to the variable-width characters used by \LaTeX ,
then the second and third columns will get moved to the right, and so, similarly,
any code indented under the other columns will be wrongly positioned.

Where a section of internal alignment coincides with the bottom of the user's page,
it can run off the bottom of the page.  This is because the {\tt tabular} environment
used for internal alignment sections does not allow pagebreaks.  Therefore the pagebreak
will come after the section has been completed.

\section{Mathematical symbols}

Mathematical symbols are always written in math font.  This means that where, say,
comments are re-defined to be in typewriter font, as in the following
example, any mathematical symbols in the comments
will appear in math font, rather than typewriter font.
\begin{quote}
\def\com{\tt}
\begin{tabbing}
{\rm -}{\rm -}\xspa{1}{\com note\/}\xspa{1}{\com that\/}\xspa{1}{\com x\/}\xspa{1}$+$\xspa{1}{\com y\/}\xspa{1}$=$\xspa{1}{\com z\/}
\end{tabbing}

\end{quote}

\section{Left indentation}

Where a line is indented beyond the end of its predecessor and aligns under another
line, but when typeset, the predecessor becomes longer than the indentation level
due to the variable-width characters,  the line's indentation will appear to be under the
predecessor line.

\section{Floating point numbers}

Currently {\tt pphs} will recognise strings such as {\tt 3.} or {\tt 5.6e} as
valid floating point numbers.  This needs rectifying so only valid floating
point numbers are recognised.