\documentclass{report}
\setlength{\oddsidemargin}{0in}
\setlength{\topmargin}{-50pt}
\setlength{\textheight}{8.5in}
\setlength{\textwidth}{6.5in}

\newcommand{\CopyrightNotice}
{Copyright \copyright{} 1998--1999 Chris Okasaki

Permission is hereby granted, free of charge, to any person obtaining a copy
of this software and associated documentation files (the "Software"), to deal
in the Software without restriction, including without limitation the rights
to use, copy, modify, merge, publish, distribute, sublicense, and/or sell
copies of the Software, and to permit persons to whom the Software is
furnished to do so, subject to the following conditions:

The above copyright notice and this permission notice shall be included in
all copies or substantial portions of the Software.

THE SOFTWARE IS PROVIDED ``AS IS'', WITHOUT WARRANTY OF ANY KIND, EXPRESS OR
IMPLIED, INCLUDING BUT NOT LIMITED TO THE WARRANTIES OF MERCHANTABILITY,
FITNESS FOR A PARTICULAR PURPOSE AND NONINFRINGEMENT.  IN NO EVENT SHALL THE
AUTHORS OR COPYRIGHT HOLDERS BE LIABLE FOR ANY CLAIM, DAMAGES OR OTHER
LIABILITY, WHETHER IN AN ACTION OF CONTRACT, TORT OR OTHERWISE, ARISING FROM,
OUT OF OR IN CONNECTION WITH THE SOFTWARE OR THE USE OR OTHER DEALINGS IN
THE SOFTWARE.
}

\title{Edison User's Guide\thanks{\CopyrightNotice} \\ (Haskell version)}
\author{Chris Okasaki}

\usepackage{xspace}

\newcommand{\ignore}[1]{}
\newcommand{\cd}{\texttt}
\newcommand{\arrow}{\ensuremath{\rightarrow}\xspace}
\newcommand{\Arrow}{\ensuremath{\Rightarrow}\xspace}
\newcommand{\Equiv}{\ensuremath{\equiv}\xspace}
\newcommand{\Implies}{\ensuremath{\Longrightarrow}\xspace}
\newcommand{\is}{:$\!$:$$\xspace}
\newcommand{\Section}[1]{Section~\ref{#1}}
\newcommand{\Chapter}[1]{Chapter~\ref{#1}}
\newcommand{\Figure}[1]{Figure~\ref{#1}}
\newcommand{\nl}{\hspace*{0pt}\\}
\newcommand{\spec}[1]{\item[\cd{\begin{tabular}{@{}l} #1 \end{tabular}}] \nl}
\newcommand{\impl}[1]{\item \texttt{#1}:}
\newcommand{\axioms}[1]{
  \par \noindent \textbf{Axioms:} \\
  \hspace*{20pt}{\ttfamily \begin{tabular}{l} #1 \end{tabular}}}
\newcommand{\eff}[1]{
  \par \noindent \textbf{Default running time:} $#1$}
\newcommand{\efftext}[1]{
  \par \noindent \textbf{Default running time:} #1}
\newcommand{\List}[1]{{[}#1{]}}
\newcommand{\AC}{\textsc{ac}\xspace}
\newcommand{\ACs}{\textsc{ac}s\xspace}
\newcommand{\hsp}{\hspace*{15pt}}

\begin{document}
\maketitle
\tableofcontents
\listoffigures

\chapter{Introduction}

Edison is a library of efficient data structures suitable for
implementation and use in functional programming languages.  
It is named after Thomas Alva Edison and for the mnemonic value of 
{\bf ED}i{\bf S}on ({\bf E}fficient {\bf D}ata {\bf S}trucutres).  
The current version of the library supports Haskell.  Future versions
of the library will also support Standard ML and possibly Scheme.

Edison provides several families of abstractions, each with multiple
implementations, along with guidance on how to choose the best
implementation for your particular application.  The main abstractions
currently supported by Edison are 
\begin{itemize} 
\item \emph{sequences} (e.g., stacks, queues, deques), 
\item \emph{collections} (e.g., sets, bags, priority queues where 
  the priority is the element), and 
\item \emph{associative collections} (e.g., finite maps, priority queues 
  where the priority and element are distinct).
\end{itemize}

Note that, in its current state, the library is mostly a framework.
That is, I provide signatures, but not yet very many implementations.
I intend to populate this framework over time, adding a new module
every few weeks.  Thus, the library is extremely unstable in the sense
that I will continually be modifying existing data structures and
adding new ones.  However, I hope that the signatures will remain
fairly stable over time, making these changes to the implementations
mostly transparent to users of the library.

If you wish to request a particular data structure or volunteer to
provide an implementation, send me email at \texttt{cdo@cs.columbia.edu}.

%%%%%%%%%%%%%%%%%%%%%%%%%%%%%%%%%%%%%%%%%%%%%%%%%%%%%%%%%%%%%%%%%%%%%%

\chapter{Installation and Use}

Download the compressed tar file and uncompress/untar it in the
location of your choice.  It will create a directory called
\texttt{edison} with several subdirectories containing the actual
source code.

Now, \cd{cd} to the \cd{edison} directory and type
\begin{verbatim}
    make all
\end{verbatim}
I recommend using GHC 4.00 or higher (I used GHC 4.00 for testing).
See \Section{portability} for a discussion of portability issues.

To compile a file \cd{Foo.hs} that uses Edison, type something like
\begin{verbatim}
    ghc -c -i/usr/local/edison/Import Foo.hs
\end{verbatim}
where \cd{/usr/local} should be replaced with the correct path
to the \cd{edison} directory.  To compile and link, type something
like
\begin{verbatim}
    ghc -i/usr/local/edison/Import -L/usr/local/edison/Lib -ledison Foo.hs
\end{verbatim}


%%%%%%%%%%%%%%%%%%%%%%%%%%%%%%%%%%%%%%%%%%%%%%%%%%%%%%%%%%%%%%%%%%%%%%

\chapter{Conventions}

In this section, I describe the general conventions I have followed
in designing Edison.  I include only those conventions that affect
users of the library.  Conventions that affect developers are described
in the separate \emph{Edison Developer's Guide}.

\section{Use of modules} \label{use-of-modules}

Each data structure is implemented as a Haskell module.  These
modules should always be imported \cd{qualified} to prevent a flood of
name clashes (see \Section{name-clashes}).  
I recommend renaming each imported module (using the
\cd{as} keyword) both to reduce the overhead of qualified names and to
make substituting one module for another as painless as possible.
The one Edison module that is typically imported unqualified is the
\cd{EdisonPrelude}.

\bigskip \noindent
Example use of an Edison data structure:
\begin{verbatim}
  module Foo where

  import EdisonPrelude
  import qualified SimpleQueue as Q

  data Tree a = Empty | Node a (Tree a) (Tree a)

  breadthFirst :: Tree a -> [a]
  breadthFirst t = bfs (Q.single t)
    where bfs q = 
            case Q.lview q of
              Just2 (Node x l r) q' -> x : bfs (Q.snoc (Q.snoc q' l) r)
              Just2 Empty q' -> bfs q'
              Nothing2 -> []
\end{verbatim}

\section{Reuse of names} \label{name-clashes}

I have attempted to choose names that are as standard as possible.
This means that operations for different abstractions frequently share
the same name (\cd{empty}, \cd{null}, \cd{size}, etc.).  It also means
that in many cases I have reused names from the Prelude.  However,
these name clashes should not be a problem because I expect Edison
modules to be imported \cd{qualified} (see \Section{use-of-modules}).
If, for some reason, you choose to import Edison modules unqualified,
you will usually need to import the Prelude \cd{hiding} the relevant
names.

Edison modules also frequently share type names.  For example, every
sequence type constructor is named \cd{Seq} unless there is a good reason
otherwise.  This makes substituting one module for another fairly
painless, especially when imported modules are renamed as in the previous
section.

An example of ``a good reason otherwise'' for not using the standard
type name is when the type constructor has the wrong kind.  This usually
happens when one data structure is parameterized by another.  For example,
an implementation of sequences that is parameterized on
another implementation of sequences might be given as
\begin{verbatim}
  data HigherOrderSeq seq a = ...
  instance Sequence seq => Sequence (HigherOrderSeq seq) where ...
\end{verbatim}
However, even modules such as these will typically define \cd{Seq}
as a type synonym for some good default choice, e.g.,
\begin{verbatim}
  type Seq = HigherOrderSeq BankersQueue.Seq
\end{verbatim}

\section{Use and non-use of classes}

Each family of abstractions is defined as a set of classes---a main
class that every implementation of that abstraction should support and
several auxiliary subclasses that an implementation may or may not support.

However, not all applications need the power of classes, so each method is 
also accessible directly from the implementation module.  For example, an 
implementation of sequences will typically define and export a 
type \cd{Seq~a} together with all the relevant functions on that type
(using the same names as the corresponding methods!), and, in addition, 
declare \cd{Seq} to be an instance of the \cd{Sequence} class.

Thus, you can choose to use overloading or not, as appropriate for your
particular application.  For example, in the following module,
\begin{verbatim}
  module Foo where
  import qualified Sequence as S
  import qualified HoodMelvilleQueue as Q
  ...
\end{verbatim}
I could refer to the empty queue of type \cd{Q.Seq~a} as either
\cd{S.empty} or \cd{Q.empty}.  The former refers to the empty method
of the \cd{Sequence} class (which might be resolved using a type
signature to be of type \cd{Q.Seq~a}) and the latter refers
directly to the empty value from \cd{Q}, without going through the
class mechanism.

Note that this example is somewhat unrealistic.
In practice, you would very rarely need to import both modules
in their entirety.  Usually, you would either import only the implementation
module
\begin{verbatim}
  import qualified HoodMelvilleQueue as Q
\end{verbatim}
or import the class module together with the type from the implementation
module
\begin{verbatim}
  import qualified Sequence as S
  import HoodMelvilleQueue (Seq)
\end{verbatim}
Note that in the last line, I imported \cd{HoodMelvilleQueue}
unqualified.  Normally, this would produce a flood of name clashes,
but it is acceptable when you are selectively importing only a type
name or two.

\section{Bonus Operations}

Some implementations export a few extra operations beyond those included
in the relevant classes.  These are typically operations that are
particularly efficient for that implementation, but which are not general
enough to warrant inclusion in a class.  An example of this is the
function \cd{unsafeMapMonotonic} that is supported by many priority
queues.

\section{Fixity}

Since qualified infix symbols are fairly ugly, I avoid infix symbols
as much as possible.  For example, I call the sequence catenation function
\cd{append} instead of \cd{++}.

\section{Error handling}

Since Haskell has no good way to recover from errors, I avoid
signalling errors if there is any reasonable alternative.  For many
functions, it is easy to avoid this by returning the \cd{Maybe} type (or
something similar), but sometimes, as with the \cd{head} function on lists
and the corresponding \cd{lhead} function on sequences, this approach is
just too painful.  For \cd{lhead} of an empty sequence, there really is no
choice but to signal an error, but other times there is a reasonable
alternative.  For example, I define both \cd{ltail} of the empty sequence
and \cd{take} of a negative argument to return the empty sequence even
though the corresponding Prelude functions would signal errors in both
of these cases.

\section{(Lack of) Portability} \label{portability}

Edison does not conform to either Haskell 1.4 or Haskell98.
It will hopefully conform to Haskell 2.  For now, it is guaranteed to
run only under GHC.

There are three non-standard language features that I use.  Of these,
only the first is critical.  The other two affect some fraction of the
individual modules, and could be eliminated fairly easily.
\begin{itemize}
\item \emph{Multi-parameter type classes.} Fortunately, this is also
  the least controversial.  Some form of multi-parameter type classes
  seems certain to make it into Haskell 2, and I expect it to be easy
  to adapt Edison's signatures accordingly.
\item \emph{Unboxed integers.} I frequently use unboxed integers when,
  for example, every node in a tree needs to maintain some sort of size
  field.  I could simply declare these fields to be strict, but
  using unboxed integers in these situations is significantly faster.
\item \emph{Pattern guards.} I occasionally use pattern guards.
  Getting by without them would not be difficult, but, since pattern guards 
  also seem likely to make it into Haskell 2, I have not bothered.
\end{itemize}
If your compiler does not support unboxed integers or pattern guards, you 
will still be able to use most of Edison---just not those 
individual data structures that use these features.
If your compiler does not support multi-parameter type classes, then
you are in much bigger trouble---you will only be able to use the
sequence data structures, not collections or associative collections.

\section{Unsafe operations}

Consider converting a list of elements into a binary search tree.
This can be implemented particularly efficiently if we know that the list is 
already sorted.  And in fact, it often is already sorted in practice.
This special case, and a handful of similar cases, are common enough and
important enough to deserve their own functions.  These are functions
with non-trivial preconditions that are too expensive to check at runtime,
so we simply make their behavior undefined if the preconditions are not
satisfied.  Violating the preconditions may break the implementations in
horrible ways, so we adopt a special naming convention to emphasize that
these operations are unsafe---almost all such functions are given names 
beginning with the word \cd{unsafe} (\cd{unsafeFromOrdList},
\cd{unsafeInsertMin}, etc.).

The one place where I have violated this convention is in the \cd{Set}
class, where there is a whole family of operations with names like
\cd{insertWith} and \cd{unionWith}.  These functions take a combining
function that is used to resolve collisions.  For example, when
inserting an element into a set that already contains that element,
the combining function is called on the new and old elements to
determine which element will remain in the new set.\footnote{
  Such a combining function is useful only when nominally
  equal elements are distinguishable in other ways---that is, when
  the ``equality'' relation is really an equivalence relation.
  However, this is extremely common.
}  
The combining functions typically return one element or the other, but
they can also combine the elements in non-trivial ways.  These
combining functions are required to satisfy the precondition that,
given two equal elements, they return a third element that is equal to
the other two.

\section{Currying}

All types in Edison are fully curried.

\section{Order of arguments}

Whenever a function takes multiple arguments, there is a choice as to
the order of the arguments.  I have tried to make these choices according
to the following rules, in decreasing order of importance:
\begin{itemize}
\item \emph{Favor an order that is more useful for partial applications.}
For example, the \cd{member} function
\begin{verbatim}
  member :: CollX c a => c a -> a -> Bool
\end{verbatim}
takes the collection first and the element second, rather than the other
way around, because it is much more commonly partially applied to a
collection than to an element.

\item \emph{Favor an order with significant mnemonic value.}
For example, the \cd{cons} and \cd{snoc} functions
\begin{verbatim}
  cons :: Sequence s => a -> s a -> s a
  snoc :: Sequence s => s a -> a -> s a
\end{verbatim}
take their arguments in opposite orders because \cd{cons} adds an
element to the left of a sequence and \cd{snoc} adds an element to the
right of a sequence.

\item \emph{Functions that modify a collection should take the collection last.}
For example, the \cd{insert} function has type
\begin{verbatim}
  insert :: CollX c a => a -> c a -> c a
\end{verbatim}
Taking the collection last supports a convenient style of stringing several
update operations in a row using the \cd{\$} combinator, as in
\begin{verbatim}
  insert 1 $ insert 2 $ insert 3 ns
\end{verbatim}
rather than
\begin{verbatim}
  insert (insert (insert ns 3) 2) 1
\end{verbatim}

\item \emph{Consistency with similar operations.}
\item \emph{Personal taste.}
\end{itemize}

\section{Completeness}

In designing a library, there is always a delicate question of how much
to put in, and how much to leave out.  Including too much can lead to code
bloat, and make the library somewhat harder to learn.  Including too little
can make the library significantly less useful.  I have tried to err on the
side of including too much, rather than too little.

Note that this can have an affect on efficiency.  Because the classes
in Edison have many methods, the dictionaries for the classes will be
large, and so building these dictionaries dynamically will be
expensive.  Fortunately, most dictionaries can be built statically.
The exceptions often involve things like non-regular datatypes and
polymorphic recursion.  This is not to say don't use these features;
just don't mix them with overloading and expect the result to be
efficient.

\section{(In)Efficiency}

In text books on data structures and algorithms, data structures are
often organized into hierarchies according to efficiency.  For
example, a hierarchy of sequences might include such entries as
\begin{itemize}
\item \emph{queues} support efficicent insertions at the rear and 
  deletions from the front
\item \emph{steques} (stack-ended queues, also known as
  output-restricted deques) are queues that additionally support
  efficient insertions at the front
\item \emph{deques} (double-ended queues) are steques that
  additionally support efficient deletions from the back
\end{itemize}
Unfortunately, a class hierarchy structured along these lines
is so fine-grained that it is nearly impossible to use.

In Edison, the class hierarchies are determined by functionality, not
by efficiency.  For example, the hierarchy for sequences contains only
a single class \cd{Sequence}, because all the sequence operations are
possible on all the sequence implementations, even if some of the
operations are inefficient for some of the implementations.
Similarly, the root class for collections--which include sets, bags,
and priority queues--contains a \cd{member} method, even though this
function is rather inefficient for most priority queues.

At some later date, we may support a special mode that prints out a
warning whenever one of the particularly inefficient operations is called.
This would be implemented by replacing the relevant default definitions
(the inefficient operations almost always just use one of the defaults)
with calls to a warning function, such as
\begin{verbatim}
  foo = warn moduleName "foo" fooDefault
\end{verbatim}
instead of
\begin{verbatim}
  foo = fooDefault
\end{verbatim}
The \cd{warn} function would either print out a message and then return
its third argument, or simply return the third argument without printing
a message, depending perhaps on a compiler flag, or on which library you
linked in.  \emph{Let me emphasize that we do nothing like this yet.}

\section{Strictness}

Most of the operations on most of the data structures are strict.  This is
inevitable for data structures with non-trivial invariants.  Even given that,
however, many of the operations are stricter than necessary.  In fact, I never
\emph{deliberately} make any computations lazy, unless the laziness is
required by the algorithm (as often happens with amortized data structures,
for instance).  In particular, I never use irrefutable patterns to make
something lazier.

Note, however, that the various sequence implementations are always
lazy in their elements.  Similarly, associative collections are
always lazy in their elements (but usually strict in their keys).

%%%%%%%%%%%%%%%%%%%%%%%%%%%%%%%%%%%%%%%%%%%%%%%%%%%%%%%%%%%%%%%%%%%%%%

\chapter{The Edison Prelude}

The \cd{EdisonPrelude} module defines a few widely-used
utility types and classes.  It is imported by every other module 
in Edison, and by most clients.  (It would be imported by all clients,
except that many modules re-export the relevant prelude entries.)
Currently, the \cd{EdisonPrelude} defines two types in the
\cd{Maybe} family, and three classes related to hashing.
%
\begin{verbatim}
  data Maybe2 a b = Just2 a b | Nothing2
  data Maybe3 a b c = Just3 a b c | Nothing3

  class Eq a => Hash a where
    hash :: a -> Int
    -- forall x,y :: a. (x == y) implies (hash x == hash y)

  class Hash a => UniqueHash a
    -- no new methods, just a stronger invariant
    -- forall x,y :: a. (x == y) iff (hash x == hash y)

  class UniqueHash a => ReversibleHash a where
    unhash :: Int -> a
    -- forall x :: a. unhash (hash x) == x
\end{verbatim}
%
The \cd{Maybe2} and \cd{Maybe3} types are used as the return types of 
functions that destructure a container, returning the element (or a key and 
element) together with the remaining container.
The \cd{Hash} classes are used to build functional analogs to
traditional imperative hash tables.

This module will gradually be expanded to include utility functions
on the \cd{Maybe2} and \cd{Maybe3} types, and perhaps a few other 
widely-used types or functions (such as \cd{warn}).

%%%%%%%%%%%%%%%%%%%%%%%%%%%%%%%%%%%%%%%%%%%%%%%%%%%%%%%%%%%%%%%%%%%%%%

\chapter{Sequences}

The \emph{sequence} abstraction is usually viewed as a hierarchy of
ADTs including lists, queues, deques, catenable lists, etc.  However,
such a hierarchy is based on efficiency rather than functionality.
For example, a list supports all the operations that a deque supports,
even though some of the operations may be inefficient.  Hence, in
Edison, all sequence data structures are defined as instances of
the single \cd{Sequence} class:
\begin{verbatim}
  class (Functor s, MonadPlus s) => Sequence s
\end{verbatim}
All sequences are also instances of \cd{Functor}, \cd{Monad},
and \cd{MonadPlus}.  In addition, all sequences
are expected to be instances of \cd{Eq} and \cd{Show},
although this is not enforced (in fact, is not enforceable in any
reasonable way).

We follow the naming convention that every module implementing sequences
defines a type constructor named \cd{Seq}.

\Figure{seq-methods} summarizes all the methods on sequences.  We next
describe each of these methods in more detail.

\begin{figure}
\begin{center}
\large\bfseries Sequence Methods
\end{center}
\hrule
\begin{description}
\item[Constructors:] \nl
    \cd{empty}, \cd{single}, \cd{cons}, \cd{snoc}, \cd{append}, \cd{fromList}, \cd{copy}, \cd{tabulate}

\item[Destructors:] \nl
    \cd{lview}, \cd{lhead}, \cd{ltail}, \cd{rview}, \cd{rhead}, \cd{rtail}

\item[Observers:] \nl
    \cd{null}, \cd{size}, \cd{toList}

\item[Concat and reverse:] \nl
    \cd{concat}, \cd{reverse}, \cd{reverseOnto}

\item[Maps and folds:] \nl
    \cd{map}, \cd{concatMap}, \cd{foldr}, \cd{foldl}, \cd{foldr1}, \cd{foldl1}, \cd{reducer}, \cd{reducel}, \cd{reduce1}

\item[Subsequences:] \nl
    \cd{take}, \cd{drop}, \cd{splitAt}, \cd{subseq}

\item[Predicate-based operations:] \nl
    \cd{filter}, \cd{partition}, \cd{takeWhile}, \cd{dropWhile}, \cd{splitWhile}

\item[Index-based operations:] \nl
    \cd{inBounds}, \cd{lookup}, \cd{lookupM}, \cd{lookupWithDefault}, \cd{update}, \cd{adjust}, \\
    \cd{mapWithIndex}, \cd{foldrWithIndex}, \cd{foldlWithIndex}

\item[Zips and unzips:] \nl
    \cd{zip}, \cd{zip3}, \cd{zipWith}, \cd{zipWith3}, \cd{unzip}, \cd{unzip3}, \cd{unzipWith}, \cd{unzipWith3}
\end{description}
\hrule
\caption{Summary of methods for the \cd{Sequence} class.}
\label{seq-methods}
\end{figure}

\section{Constructors}
\begin{description}
\spec{empty \is seq a}
  The empty sequence.
\spec{single \is a \arrow seq a}
  Create a singleton sequence.
  \axioms{single x \Equiv cons x empty \Equiv snoc empty x}
  \eff{O(1)}
\spec{cons \is a \arrow seq a \arrow seq a}
  Add a new element to the front/left of a sequence.
  \axioms{cons x xs \Equiv append (single x) xs}
  \eff{O(1)}
\spec{snoc \is seq a \arrow a \arrow seq a}
  Add a new element to the rear/right of a sequence.
  \axioms{snoc xs x \Equiv append xs (single x)} 
  \eff{O(n)}
\spec{append \is seq a \arrow seq a \arrow seq a}
  Append two sequences, with the first argument on the left and
  the second argument on the right.
  \axioms{append xs ys \Equiv foldr cons ys xs}
  \eff{O(n_1)}
\spec{fromList \is \List{a} \arrow seq a}
  Convert a list to a sequence.
  \axioms{fromList xs \Equiv foldr cons empty xs}
  \eff{O(n)}
\spec{copy \is Int \arrow a \arrow seq a}
  Create a sequence containing $n$ copies of the given element.
  Return \cd{empty} if $n<0$.
  \axioms{
    n > 0 \Implies copy n x \Equiv cons x (copy (n-1) x) \\
    n <= 0 \Implies copy n x \Equiv empty}
  \eff{O(n)}
\spec{tabulate \is Int \arrow (Int \arrow a) \arrow seq a}
  Create a sequence containing the results of applying the given
  function to the integers $0\ldots n-1$.
  Return \cd{empty} if $n<0$.
  \axioms{
    n > 0 \Implies tabulate n f \Equiv map f (fromList [0..n-1]) \\
    n <= 0 \Implies tabulate n f \Equiv empty}
  \efftext{$O(nt)$, where $t$ is the running time of \cd{f}}
\end{description}

\section{Destructors}
\begin{description}
\spec{lview \is seq a \arrow Maybe2 a (seq a)}
  Separate a sequence into its first element and the remaining sequence.
  Return \cd{Nothing2} if the sequence is empty.
  \axioms{lview empty \Equiv Nothing2 \\
          lview (cons x xs) \Equiv Just2 x xs}
  \eff{O(1)}
\spec{lhead \is seq a \arrow a}
  Return the first element of the sequence.  Signal an error if the
  sequence is empty.
  \axioms{lhead empty \Equiv \emph{error} \\
          lhead (cons x xs) \Equiv x}
  \eff{O(1)}
\spec{ltail \is seq a \arrow seq a}
  Delete the first element of the sequence.  Return \cd{empty} if
  the sequence is already empty.
  \axioms{ltail empty \Equiv empty \\
          ltail (cons x xs) \Equiv xs}
  \eff{O(1)}
\spec{rview \is seq a \arrow Maybe2 (seq a) a}
  Separate a sequence into its last element and the remaining sequence.
  Return \cd{Nothing2} if the sequence is empty.
  \axioms{rview empty \Equiv Nothing2 \\
          rview (snoc xs x) \Equiv Just2 xs x}
  \eff{O(n)}
\spec{rhead \is seq a \arrow a}
  Return the first element of the sequence.  Signal an error if the
  sequence is empty.
  \axioms{rhead empty \Equiv \emph{error} \\
          rhead (snoc xs x) \Equiv x}
  \eff{O(n)}
\spec{rtail \is seq a \arrow seq a}
  Delete the first element of the sequence.  Return \cd{empty} if
  the sequence is already empty.
  \axioms{rtail empty \Equiv empty \\
          rtail (snoc xs x) \Equiv xs}
  \eff{O(n)}
\end{description}

\section{Observers}
\begin{description}
\spec{null \is seq a \arrow Bool}
  Return \cd{True} if the sequence is empty and \cd{False} otherwise.
  \axioms{null xs \Equiv (size xs == 0)}
  \eff{O(1)}
\spec{size \is seq a \arrow Int}
  Return the length of the sequence.
  \axioms{size empty \Equiv 0 \\
          size (cons x xs) \Equiv 1 + size xs}
  \eff{O(n)}
\spec{toList \is seq a \arrow \List{a}}
  Convert a sequence to a list.
  \axioms{toList empty \Equiv [] \\
          toList (cons x xs) \Equiv x :$$ toList xs}
  \eff{O(n)}
\end{description}

\section{Concat and reverse}
\begin{description}
\spec{concat \is seq (seq a) \arrow seq a}
  Flatten a sequence of sequences into a simple sequence.
  \axioms{concat xss \Equiv foldr append empty xss}
  \efftext{$O(n + m)$, where $n$ is the length of the input sequence
    and $m$ is the length of the output sequence (usually $n < m$, but
    if the input sequence contains many empties, then $n$ may be larger)}
\spec{reverse \is seq a \arrow seq a}
  Reverse the order of a sequence.
  \axioms{reverse empty \Equiv empty \\
          reverse (cons x xs) \Equiv snoc (reverse xs) x}
  \eff{O(n)}
\spec{reverseOnto \is seq a \arrow seq a \arrow seq a}
  Reverse a sequence onto the front of another sequence.
  \axioms{reverseOnto xs ys \Equiv append (reverse xs) ys}
  \eff{O(n_1)}
\end{description}

\section{Maps and folds}
\begin{description}
\spec{map \is (a \arrow b) \arrow seq a \arrow seq b}
  Return the result of applying a function to every element of a sequence.
  \axioms{map f empty \Equiv empty \\
          map f (cons x xs) \Equiv cons (f x) (map f xs)}
  \efftext{$O(nt)$, where $t$ is the running time of \cd{f}}
\spec{concatMap \is (a \arrow seq b) \arrow seq a \arrow seq b}
  Apply a sequence-producing function to every element of a sequence
  and flatten the result.  Note that \cd{concatMap} is
  the ``bind'' operation of the sequence monad (but with the arguments
  in the opposite order).
  \axioms{concatMap f xs \Equiv concat (map f xs)}
  \efftext{$O(nt + m)$, where $n$ is the length of the input sequence,
    $m$ is the length of the output sequence, and $t$ is the running
    time of \cd{f}}
\spec{foldr \is (a \arrow b \arrow b) \arrow b \arrow seq a \arrow b}
  Combine all the elements of a sequence into a single value, given
  a right-associative combining function and an initial value.
  Note that 
\[\cd{foldr}\; (\oplus)\; \epsilon\; [x_0,x_1,\ldots,x_{n-1}] \;\equiv\;
  x_0 \oplus (x_1 \oplus \cdots \oplus (x_{n-1} \oplus \epsilon))
\]
  \axioms{foldr f c empty \Equiv c \\
          foldr f c (cons x xs) = f x (foldr f c xs)}
  \efftext{$O(nt)$, where $t$ is the running time of \cd{f}}
\spec{foldl \is (b \arrow a \arrow b) \arrow b \arrow seq a \arrow b}
  Combine all the elements of a sequence into a single value, given
  a left-associative combining function and an initial value.
  Note that 
\[\cd{foldl}\; (\oplus)\; \epsilon\; [x_0,x_1,\ldots,x_{n-1}] \;\equiv\;
    ((\epsilon \oplus x_0) \oplus x_1) \oplus \cdots \oplus x_{n-1}
\]
  \axioms{foldl f c empty \Equiv c \\
          foldl f c (cons x xs) = foldl f (f c x) xs}
  \efftext{$O(nt)$, where $t$ is the running time of \cd{f}}
\spec{foldr1 \is (a \arrow a \arrow a) \arrow seq a \arrow seq a}
  Combine all the elements of a non-empty sequence into a single value,
  given a right-associative combining function.  Signal an error if the
  sequence is empty.
  \axioms{foldr1 f empty \Equiv \emph{error} \\
          foldr1 f (snoc xs x) \Equiv foldr f x xs}
  \efftext{$O(nt)$, where $t$ is the running time of \cd{f}}
\spec{foldl1 \is (a \arrow a \arrow a) \arrow seq a \arrow seq a}
  Combine all the elements of a non-empty sequence into a single value,
  given a left-associative combining function.  Signal an error if the
  sequence is empty.
  \axioms{foldl1 f empty \Equiv \emph{error} \\
          foldl1 f (cons x xs) \Equiv foldl f x xs}
  \efftext{$O(nt)$, where $t$ is the running time of \cd{f}}

\spec{reducer \is (a \arrow a \arrow a) \arrow a \arrow seq a \arrow seq a \\
      reducel \is (a \arrow a \arrow a) \arrow a \arrow seq a \arrow seq a \\
      reduce1 \is (a \arrow a \arrow a) \arrow seq a \arrow seq a}
  Like the various folds, but combine the elements in a balanced fashion
  rather than linearly from right-to-left or left-to-right.  Usually,
  the combining function is associative, in which case the various
  reduces yield the same answers as the corresponding folds (albeit perhaps
  more efficiently).  \cd{reduce1} signals an error if the
  sequence is empty.

  What do I mean by ``in a balanced fashion''?  I mean that
  $\cd{reduce1}\; (\oplus)\; [x_0,\ldots,x_{n-1}]$ equals some
  complete parenthesization of $x_0 \oplus \cdots \oplus x_{n-1}$,
  such that the nesting depth of parentheses is $O(\log n)$.  The
  precise shape of this parenthesization is unspecified.  For example,
  the following are all typical answers for
  $\cd{reduce1}\; (\oplus)\; [a,b,c,d,e,f]$:
\[\begin{array}{l}
  (a \oplus b) \oplus ((c \oplus d) \oplus (e \oplus f)) \\
  ((a \oplus b) \oplus (c \oplus d)) \oplus (e \oplus f) \\
  (a \oplus (b \oplus c)) \oplus (d \oplus (e \oplus f)) \\
  ((a \oplus b) \oplus c) \oplus ((d \oplus e) \oplus f)
  \end{array}\]
  \emph{Note that these are the only sequence operations
  for which different implementations are permitted to yield different 
  answers.}\footnote{
    For all the other sequence operations, implementations
    may differ only in efficiency and strictness/order of evaluation.}
  Also note that a single implementation may choose different parenthesizations
  for different sequences, even if they are the same length.  This will 
  typically happen when the lists were constructed differently (e.g.,
  one using \cd{cons} and the other using \cd{snoc}).

  The canonical applications of the \cd{reduce} functions are algorithms
  like \cd{mergesort}, where
\begin{verbatim}
    mergesort :: (Ord a,Sequence s) => s a -> s a
    mergesort xs = reducer merge empty (map single xs)
\end{verbatim}

  \axioms{
    reduce1 ($\oplus$) empty \Equiv \emph{error} \\
    $\forall x,y,z.\; x \oplus (y \oplus z) \Equiv (x \oplus y) \oplus z$ \ \Implies \\
    \hspace*{20pt} reduce1 ($\oplus$) xs \Equiv foldr1 ($\oplus$) xs 
                                         \Equiv foldl1 ($\oplus$) xs \\
    \hspace*{20pt} reducer ($\oplus$) c xs \Equiv foldr ($\oplus$) c xs \\
    \hspace*{20pt} reducel ($\oplus$) c xs \Equiv foldl ($\oplus$) c xs}
  \efftext{$O(nt)$, where $t$ is the running time of \cd{f}}
\end{description}

\section{Subsequences}
\begin{description}
\spec{take \is Int \arrow seq a \arrow seq a}
  Extract a prefix of length $i$ from a sequence.
  Return \cd{empty} if $i$ is negative, or the entire sequence if $i$
  is too large.
  \axioms{i < 0 \Implies take i xs \Equiv empty \\
          i > size xs \Implies take i xs \Equiv xs \\
          size xs == i \Implies take i (append xs ys) \Equiv xs}
  \eff{O(i)}
\spec{drop \is Int \arrow seq a \arrow seq a}
  Delete a prefix of length $i$ from a sequence.
  Return the entire sequence if $i$ is negative, or \cd{empty} if
  $i$ is too large.
  \axioms{i < 0 \Implies drop i xs \Equiv xs \\
          i > size xs \Implies drop i xs \Equiv empty \\
          size xs == i \Implies drop i (append xs ys) \Equiv ys}
  \eff{O(i)}
\spec{splitAt \is Int \arrow seq a \arrow (seq a, seq a)}
  Split a sequence into a prefix of length $i$ and the remaining
  sequence.  Behaves the same as the corresponding calls to \cd{take}
  and \cd{drop} is $i$ is negative or too large.
  \axioms{splitAt i xs \Equiv (take i xs, drop i xs)}
  \eff{O(i)}
\spec{subseq \is Int \arrow Int \arrow seq a \arrow seq a}
  Extract a subsequence from a sequence.  The integer arguments are 
  ``start index'' and ``length'' rather than ``start index'' and ``end index''.
  Behaves the same as the corresponding calls to \cd{take} and \cd{drop}
  if the start index or length are negative or too large.
  \axioms{subseq i len xs \Equiv take len (drop i xs)}
  \eff{O(i + len)}
\end{description}

\section{Predicate-based operations}
\begin{description}
\spec{filter \is (a \arrow Bool) \arrow seq a \arrow seq a}
  Extract the elements of a sequence that satisfy the given predicate, retaining
  the relative ordering of elements from the original sequence.
  \axioms{filter p empty \Equiv empty \\
          filter p (cons x xs) \Equiv if p x then cons x (filter p xs) else filter p xs}
  \efftext{$O(nt)$, where $t$ is the running time of \cd{p}}
\spec{partition \is (a \arrow Bool) \arrow seq a \arrow (seq a, seq a)}
  Separate the elements of a sequence into those that satisfy the given
  predicate and those that do not, retaining the relative ordering of elements
  from the original sequence.
  \axioms{partition p xs \Equiv (filter p xs, filter (not .\ p) xs)}
  \efftext{$O(nt)$, where $t$ is the running time of \cd{p}}
\spec{takeWhile \is (a \arrow Bool) \arrow seq a \arrow seq a}
  Extract the maximal prefix of elements satisfying the given predicate.
  \axioms{takeWhile p empty \Equiv empty \\
          takeWhile p (cons x xs) \Equiv if p x then cons x (takeWhile p xs) else empty}
  \efftext{$O(nt)$, where $t$ is the running time of \cd{p}}
\spec{dropWhile \is (a \arrow Bool) \arrow seq a \arrow seq a}
  Delete the maximal prefix of elements satisfying the given predicate.
  \axioms{dropWhile p empty \Equiv empty \\
          dropWhile p (cons x xs) \Equiv if p x then dropWhile p xs else cons x xs}
  \efftext{$O(nt)$, where $t$ is the running time of \cd{p}}
\spec{splitWhile \is (a \arrow Bool) \arrow seq a \arrow seq a}
  Split a sequence into the maximal prefix of elements satisfying the
  given predicate, and the remaining sequence.
  \axioms{splitWhile p xs \Equiv (takeWhile p xs, dropWhile p xs)}
  \efftext{$O(nt)$, where $t$ is the running time of \cd{p}}
\end{description}

\section{Index-based operations}

The following operations all assume zero-based indexing.

\begin{description}
\spec{inBounds \is seq a \arrow Int \arrow Bool}
  Test whether an index is valid for the given sequence.
  \axioms{inBounds xs i \Equiv (0 <= i \&\& i < size xs)}
  \eff{O(i)}
\spec{lookup \is seq a \arrow Int \arrow a}
  Return the element at the given index.  Signal an error if the
  index is out of bounds.
  \axioms{not (inBounds xs i) \Implies lookup xs i \Equiv \emph{error} \\
          size xs == i \Implies lookup (append xs (cons x ys)) i \Equiv x}
  \eff{O(i)}
\spec{lookupM \is seq a \arrow Int \arrow Maybe a}
  Return \cd{Just} of the element at the given index, or \cd{Nothing}
  if the index is out of bounds.
  \axioms{not (inBounds xs i) \Implies lookupM xs i \Equiv Nothing \\
          size xs == i \Implies lookupM (append xs (cons x ys)) i \Equiv Just x}
  \eff{O(i)}
\spec{lookupWithDefault \is a \arrow seq a \arrow Int \arrow a}
  Return the element at the given index, or the default argument
  if the index is out of bounds.
  \axioms{not (inBounds xs i) \Implies lookupWithDefault d xs i \Equiv d \\
          size xs == i \Implies lookupWithDefault d (append xs (cons x ys)) i \Equiv x}
  \eff{O(i)}
\spec{update \is Int \arrow a \arrow seq a \arrow seq a}
  Replace the element at the given index, or return the original sequence
  if the index is out of bounds.
  \axioms{not (inBounds xs i) \Implies update i y xs \Equiv xs \\
          size xs == i \Implies update i y (append xs (cons x ys)) \Equiv append xs (cons y ys)}
  \eff{O(i)}
\spec{adjust \is (a \arrow a) \arrow Int \arrow seq a \arrow seq a}
  Apply a function to the element at the given index, or return the original
  sequence if the index is out of bounds.
  \axioms{not (inBounds xs i) \Implies adjust f i xs \Equiv xs \\
          size xs == i \Implies adjust f i (append xs (cons x ys)) \Equiv append xs (cons (f x) ys)}
  \efftext{$O(i + t)$, where $t$ is the running time of \cd{f}}
\spec{mapWithIndex \is (Int \arrow a \arrow b) \arrow seq a \arrow seq b}
  Like \cd{map}, but include the index with each element.
  \axioms{mapWithIndex f empty \Equiv empty \\
          mapWithIndex f (snoc xs x) \Equiv snoc (mapWithIndex f xs) (f (size xs) x)}
  \efftext{$O(nt)$, where $t$ is the running time of \cd{f}}
\spec{foldrWithIndex \is (Int \arrow a \arrow b \arrow b) \arrow b \arrow seq a \arrow b}
  Like \cd{foldr}, but include the index with each element.
  \axioms{foldrWithIndex f c empty \Equiv c \\
          foldrWithIndex f c (snoc xs x) \Equiv foldrWithIndex f (f (size xs) x c) xs}
  \efftext{$O(nt)$, where $t$ is the running time of \cd{f}}
\spec{foldlWithIndex \is (b \arrow Int \arrow a \arrow b) \arrow b \arrow seq a \arrow b}
  Like \cd{foldl}, but include the index with each element.
  \axioms{foldlWithIndex f c empty \Equiv c \\
          foldlWithIndex f c (snoc xs x) \Equiv f (foldlWithIndex f c xs) (size xs) x}
  \efftext{$O(nt)$, where $t$ is the running time of \cd{f}}
\end{description}

\section{Zips and unzips}
\begin{description}
\spec{zip \is seq a \arrow seq b \arrow seq (a,b) \\
      zip3 \is seq a \arrow seq b \arrow seq c \arrow seq (a,b,c)}
  Combine two (or three) sequences into a sequence of pairs (or triples).
  If the sequences are of different lengths, the excess elements of the
  longer sequence (or sequences) are discarded.
  \axioms{zip xs ys \Equiv zipWith ($\lambda$ x y \arrow (x,y)) xs ys \\
          zip3 xs ys zs \Equiv zipWith3 ($\lambda$ x y z \arrow (x,y,z)) xs ys zs}
  \eff{O(\min \{n_1,n_2\}), O(\min \{n_1,n_2,n_3\})}
\spec{zipWith \is (a \arrow b \arrow c) \arrow seq a \arrow seq b \arrow seq c \\
      zipWith3 \is (a \arrow b \arrow c \arrow d) \arrow seq a \arrow seq b \arrow seq c \arrow seq d}
  Combine two (or three) sequences into a single sequence by mapping
  a combining function across corresponding elements.
  If the sequences are of different lengths, the excess elements of the
  longer sequence (or sequences) are discarded.
  \axioms{zipWith f (cons x xs) (cons y ys) \Equiv cons (f x y) (zipWith f xs ys) \\
          (null xs $\vee$ null ys) \Implies zipWith f xs ys \Equiv empty \\
          zipWith3 f (cons x xs) (cons y ys) (cons z zs) \Equiv cons (f x y z) (zipWith3 f xs ys zs) \\
          (null xs $\vee$ null ys $\vee$ null zs) \Implies zipWith3 f xs ys zs \Equiv empty}
  \efftext{$O(t \cdot \min \{n_1,n_2\}),
           O(t \cdot \min \{n_1,n_2,n_3\})$, where $t$ is the running time of \cd{f}}
\spec{unzip \is seq (a,b) \arrow (seq a, seq b) \\
      unzip3 \is seq (a,b,c) \arrow (seq a, seq b, seq c)}
  Transpose a sequence of pairs (or triples) into a pair (or triple) of
  sequences.
  \axioms{unzip xys \Equiv (map fst xys, map snd xys) \\
          unzip3 xyzs \Equiv (map fst3 xyzs, map snd3 xyzs, map thd3 xyzs)}
  \eff{O(n)}
\spec{unzipWith \is (a \arrow b) \arrow (a \arrow c) \arrow seq a \arrow (seq b, seq c) \\
      unzipWith3 \is (a \arrow b) \arrow (a \arrow c) \arrow (a \arrow d) \arrow seq a \arrow (seq b, seq c, seq d)}
  Map two (or three) functions across every element of a sequence, yielding
  a pair (or triple) of sequences.
  \axioms{unzipWith f g xs \Equiv (map f xs, map g xs) \\
          unzipWith3 f g h xs \Equiv (map f xs, map g xs, map h xs)}
  \efftext{$O(nt)$, where $t$ is the maximum running time of \cd{f}, \cd{g}, and \cd{h}}
\end{description}

\section{Implementations}

The following implementations are available or planned.
I list with each implementation the major operations whose running times 
differ from the default (either better or worse).

\begin{description}
\item[Available:] \nl \vspace*{-15pt}
\begin{itemize}
\impl{ListSeq} Ordinary lists.
\impl{SimpleQueue} Burton, IPL'82. \\
     \hsp $O(1)$ \cd{snoc}. \\
     \hsp $O(1)$ \cd{lview/ltail} if single-threaded, $O(n)$ otherwise.
\impl{BankersQueue} Okasaki, JFP'95. \\
     \hsp $O(1)$ \cd{snoc}.
\impl{MyersStack} Myers, IPL'93. \\
     \hsp $O(\log n)$ \cd{lookup}.
\impl{RandList} Okasaki, FPCA'95. \\
     \hsp $O(\log n)$ \cd{lookup}, \cd{update}.
\impl{BinaryRandList} Okasaki, PFDS (Chapter 10.1.2). \\
     \hsp $O(\log n)$ \cd{lookup}, \cd{update}.
\impl{JoinList} \\
     \hsp $O(1)$ \cd{snoc/append}. \\
     \hsp $O(n)$ \cd{lview/ltail/rview/rtail}, but $O(1)$ in practice.
\end{itemize}

\item[Planned:] \nl \vspace*{-15pt}
\begin{itemize}
\impl{StrictList} Strict lists.
\impl{BootstrappedQueue} Okasaki, PFDS. \\
     \hsp $O(1)$ \cd{snoc}.
\impl{SimpleDeque} Hoogerwoord, JFP'92. \\
     \hsp $O(1)$ \cd{snoc}. \\
     \hsp $O(1)$ \cd{lview/ltail/rview/rtail} if single-threaded, 
          $O(n)$ otherwise.
\impl{CatenableList} Okasaki, FOCS'95. \\
     \hsp $O(1)$ \cd{snoc}, \cd{append}.
\impl{CatenableDeque} Okasaki, ICFP'97. \\
     \hsp $O(1)$ \cd{snoc}, \cd{rview/rhead/rtail}, \cd{append}.
\impl{BraunSeq} Hoogerwoord, MPC'92. \\
     \hsp $O(\log n)$ \cd{cons/lview/ltail}. \\
     \hsp $O(\log^2 n)$ \cd{snoc/rview/rhead/rtail}. \\
     \hsp $O(\log i)$ \cd{lookup}, \cd{update}.
\end{itemize}
\end{description}

%%%%%%%%%%%%%%%%%%%%%%%%%%%%%%%%%%%%%%%%%%%%%%%%%%%%%%%%%%%%%%%%%%%%%%

\chapter{Collections} \label{sec:coll}

The \emph{collection} abstraction includes sets, bags, and priority
queues (heaps).  Collections are defined in Edison as a set of eight
classes, organized in the hierarchy shown in \Figure{collection-hierarchy}.
These classes make essential use of multi-parameter type classes.
All collections assume at least an equality relation on elements, and
many also assume an ordering relation.  The use of multi-parameter type
classes allows any particular instance to assume further properties
as necessary (such as hashability).

\begin{figure}
\small
\begin{center}
\begin{picture}(200,200)(0,0)
\put(100,200){\makebox(0,0){\cd{Eq $a$}}}
\put(20,150){\makebox(0,0){\cd{Ord $a$}}}
\put(100,168){\line(0,1){24}}
\put(100,160){\makebox(0,0){\cd{CollX $c$ $a$}}}
\put(20,118){\line(0,1){24}}
\put(26,118){\line(2,1){68}}
\put(20,110){\makebox(0,0){\cd{OrdCollX $c$ $a$}}}
\put(174,118){\line(-2,1){68}}
\put(180,110){\makebox(0,0){\cd{SetX $c$ $a$}}}
\put(106,68){\line(2,1){68}}
\put(94,68){\line(-2,1){68}}
\put(100,60){\makebox(0,0){\cd{OrdSetX $c$ $a$}}}
\put(100,108){\line(0,0){44}}
\put(100,100){\makebox(0,0){\cd{Coll $c$ $a$}}}
\put(20,58){\line(0,0){44}}
\put(26,58){\line(2,1){68}}
\put(20,50){\makebox(0,0){\cd{OrdColl $c$ $a$}}}
\put(180,58){\line(0,0){44}}
\put(174,58){\line(-2,1){68}}
\put(180,50){\makebox(0,0){\cd{Set $c$ $a$}}}
\put(106,8){\line(2,1){68}}
\put(94,8){\line(-2,1){68}}
\put(100,0){\makebox(0,0){\cd{OrdSet $c$ $a$}}}
\end{picture} \\[40pt]

\ttfamily
\begin{tabular}{cccc}
CollX        & OrdCollX        & SetX       & OrdSetX             \\
\hline
empty,insert & deleteMin       & intersect  & \textit{no methods} \\
union,delete & unsafeInsertMin & difference &                     \\
null,size    & filterLT        & subset     &                     \\
member,count & $\cdots$        & subsetEq   &                     \\
$\cdots$     &                 &            &                     \\[20pt]

Coll         & OrdColl         & Set        & OrdSet              \\
\hline       
toSeq        & minElem         & insertWith & \textit{no methods} \\
lookup       & foldr,foldl     & unionWith  &                     \\
fold         & toOrdSeq        & intersectWith &                  \\
filter       & $\cdots$        & $\cdots$   &                     \\
$\cdots$     &                 &            &                     \\
\end{tabular}
\end{center}

\normalsize

\caption{The collection class hierarchy, with typical methods for each class.}
\label{collection-hierarchy}
\end{figure}

The hierarchy contains a root class, \cd{CollX}, together with seven
subclasses satisfying one or more of three common sub-properties:
%
\begin{itemize}
\item \emph{Uniqueness}.  Each element in the collection is unique (i.e.,
  no two elements in the collection are equal).  These subclasses,
  indicated by the name \cd{Set}, represent sets rather than bags.
\item \emph{Ordering}.  The elements have a total ordering and it is possible 
  to process the elements in non-decreasing order.  These subclasses,
  indicated by the \cd{Ord} prefix, typically represent either priority 
  queues (heaps) or sets/bags implemented as binary search trees.
\item \emph{Observability}.  An observable collection is one in which it is
  possible to view the elements in a collection.  The \cd{X} suffix
  indicates lack of observability.  This property is discussed in
  greater detail below in \Section{observability}.
\end{itemize}

Because collections encompass a wide range of abstractions, there is
no single name that is suitable for all collection type constructors.
However, most modules implementing collections will define a type constructor
named either \cd{Bag}, \cd{Set}, or \cd{Heap}.

\Figure{coll-methods} summarizes all the methods on collections.
These methods will be described in more detail in the sections on
each subclass in the hierarchy.

\begin{figure}
\begin{center}
\large\bfseries Collection Methods
\end{center}
\hrule
\begin{description}
\item[Constructors:] \nl
  \cd{CollX}: 
    \cd{empty}, \cd{single}, \cd{insert}, \cd{insertSeq}, \cd{union}, \cd{unionSeq}, \cd{fromSeq} \\
  \cd{OrdCollX}: \cd{unsafeInsertMin}, \cd{unsafeInsertMax}, \cd{unsafeFromOrdSeq}, \cd{unsafeAppend} \\
  \cd{Set}: \cd{insertWith}, \cd{insertSeqWith}, \cd{unionl}, \cd{unionr}, \cd{unionWith}, \cd{unionSeqWith}, \cd{fromSeqWith}
\item[Destructors:] \nl
  \cd{OrdColl}: \cd{minView}, \cd{minElem}, \cd{maxView}, \cd{maxElem} 
\item[Deletions:] \nl
  \cd{CollX}: \cd{delete}, \cd{deleteAll}, \cd{deleteSeq} \\
  \cd{OrdCollX}: \cd{deleteMin}, \cd{deleteMax}
\item[Observers:] \nl
  \cd{CollX}: \cd{null}, \cd{size}, \cd{member}, \cd{count} \\
  \cd{Coll}: \cd{lookup}, \cd{lookupM}, \cd{lookupAll}, \cd{lookupWithDefault}, \cd{toSeq} \\
  \cd{OrdColl}: \cd{toOrdSeq}
\item[Filters and partitions:] \nl
  \cd{OrdCollX}:
  \cd{filterLT}, \cd{filterLE}, \cd{filterGT}, \cd{filterGE}, \\
  \hspace*{20pt}
  \cd{partitionLT\_GE}, \cd{partitionLE\_GT}, \cd{partitionLT\_GT} \\
  \cd{Coll}: \cd{filter}, \cd{partition} 
\item[Set operations:] \nl
  \cd{SetX}: \cd{intersect}, \cd{difference}, \cd{subset}, \cd{subsetEq} \\
  \cd{Set}: \cd{intersectWith}
\item[Folds:] \nl
  \cd{Coll}: \cd{fold}, \cd{fold1} \\
  \cd{OrdColl}: \cd{foldr}, \cd{foldl}, \cd{foldr1}, \cd{foldl1}
\end{description}
\hrule
\caption{Summary of methods for the collections classes.}
\label{coll-methods}
\end{figure}

\section{Observability} \label{observability}

Note that the equality relation defined by the \cd{Eq} class
is not necessarily true equality.  Very often it is merely
an equivalence relation, where equivalent values may be distinguishable
by other means.  For example, we might consider two binary search trees
to be equal if they contain the same elements, even if their shapes
are different.  

Because of this phenomenon, implementations of observable collections 
(i.e., collections where it is possible to inspect the elements) are
rather constrained.  Such an implementation must retain the actual
elements that were inserted.  For example, it is not possible
in general to represent an observable bag as a finite map from elements
to counts, because even if we know that a given bag contains, say, three
elements from some equivalence class, we do not necessarily know \emph{which}
three.

On the other hand, implementations of \emph{non-observable} collections
have much greater freedom to choose abstract representations of
each equivalence class.  For example, representing a bag as a finite
map from elements to counts works fine if we never need to know
\emph{which} representatives from an equivalence class are actually
present.  As another example, consider the \cd{UniqueHash} class
defined in the Edison Prelude.  If we know that the \cd{hash} function
yields a unique integer for each equivalence class, then we can
represent a collection of hashable elements simply as a collection
of integers.  With such a representation, we can still do many
useful things like testing for membership---we just can't support functions
like \cd{fold} or \cd{filter} that require the elements themselves,
rather than the hashed values.\footnote{In fact, we can even
support \cd{fold} and \cd{filter} if the hashing function is \emph{reversible},
but this is relatively uncommon.}

\section{\cd{CollX}}
\begin{verbatim}
class Eq a => CollX c a
\end{verbatim}
%
This is the root class of the hierarchy.  However, it is perfectly
adequate for many applications that use sets or bags.

\subsection{Constructors}
\begin{description}
\spec{empty \is coll a}
  The empty collection.
%  \axioms{size empty \Equiv 0 \\
%          count empty x \Equiv 0}
\spec{single \is a \arrow coll a}
  Create a singleton collection.
%  \axioms{size (single x) \Equiv 1 \\
%          count (single x) y \Equiv if x == y then 1 else 0}
\spec{insert \is a \arrow coll a \arrow coll a \\
      insertSeq \is Sequence seq \Arrow seq a \arrow coll a \arrow coll a}
  Add an element or a sequence of elements to a collection.  For sets,
  \cd{insert} keeps the new element in the case of duplicates, but
  \cd{insertSeq} keeps an unspecified element.

  See also \cd{insertWith} and \cd{insertSeqWith} in \cd{Set}.
\ignore{
  \axioms{\textrm{For bags:} \\
          \hspace*{20pt} size (insert x xs) \Equiv 1 + size xs \\
          \hspace*{20pt} count (insert x xs) y \Equiv if x == y then 1 + count xs y else count xs y \\
          \textrm{For sets:} \\
          \hspace*{20pt} size (insert x xs) \Equiv if member xs x then size xs else 1 + size xs \\
          \hspace*{20pt} count (insert x xs) y \Equiv if x == y then 1 else count xs y \\
          \textrm{For observable sets:} \\
          \hspace*{20pt} insert x xs \Equiv insert x (delete x xs)}
}
\spec{union \is coll a \arrow coll a \arrow coll a \\
      unionSeq \is Sequence seq \Arrow seq a \arrow seq (coll a) \arrow coll a}
  Merge two collections or a sequence of collections.  For sets, it
  is unspecified which element is kept in the case of duplicates.

  See also \cd{unionl}, \cd{unionr}, \cd{unionWith}, and \cd{unionSeqWith} in
  \cd{Set}.
\spec{fromSeq \is Sequence seq \Arrow seq a \arrow coll a}
  Convert a sequence of elements into a collection.  For sets, it
  is unspecified which element is kept in the case of duplicates.

  See also \cd{fromSeqWith} in \cd{Set}.
\end{description}

\subsection{Deletions}
\begin{description}
\spec{delete \is a \arrow coll a \arrow coll a \\
      deleteAll \is a \arrow coll a \arrow coll a}
  Delete a single occurrence or all occurences of the given element
  from a collection.  If the element does not appear in the collection,
  then leave the collection unchanged. For sets, these functions will be the 
  same, but for bags
  they may be different.  For \cd{delete} on bags, it is unspecified
  which of several duplicate elements is deleted.
\spec{deleteSeq \is Sequence seq \arrow seq a \arrow coll a \arrow coll a}
  Delete a single occurrence of each of the given elements from
  a collection, ignoring those elements in the sequence that do not
  appear in the collection.  For bags, there may be multiple occurrences of a
  given element in the collection, in which case it is unspecified
  which is deleted.
\end{description}

\subsection{Observers}
\begin{description}
\spec{null \is coll a \arrow Bool}
  Test whether the collection is empty.
  \axioms{null xs \Equiv (size xs == 0)}
\spec{size \is coll a \arrow Int}
  Return the number of elements in the collection.
\spec{member \is coll a \arrow a \arrow Bool}
  Test whether the given element is in the collection.
  \axioms{member xs x \Equiv (count xs x > 0)}
\spec{count \is coll a \arrow a \arrow Int}
  Count how many copies of the given element are in the collection.
\end{description}

\section{\cd{OrdCollX}}
\begin{verbatim}
class (CollX c a, Ord a) => OrdCollX c a
\end{verbatim}

\subsection{Constructors}
\begin{description}
\spec{unsafeInsertMin \is a \arrow coll a \arrow coll a \\
      unsafeInsertMax \is coll a \arrow a \arrow coll a}
  Insert an element into a collection with the precondition that the
  new element is $\leq$ or $\geq$ any existing elements.  For sets, this 
  precondition is strengthened to $<$ or $>$.
\spec{unsafeFromOrdSeq \is Sequence seq \Arrow seq a \arrow coll a}
  Convert a sequence of elements into a collection with the precondition
  that the sequence is already sorted into non-decreasing order.  For
  sets, this precondition is strengthened to increasing order.
\spec{unsafeAppend \is coll a \arrow coll a \arrow coll a}
  Merge two collections with the precondition that every element
  in the first collection is $\leq$ every element in the second collection.
  For sets, this precondition is strengthened to $<$.
\end{description}

\subsection{Deletions}
\begin{description}
\spec{deleteMin \is coll a \arrow coll a \\
      deleteMax \is coll a \arrow coll a} 
  Delete the minimum or maximum element from the collection, or return
  \cd{empty} if the collection is empty.
  If there is more than one minimum or maximum, it is unspecified which
  is deleted.

  See also \cd{minView}, \cd{minElem}, \cd{maxView}, and \cd{maxElem}
  in \cd{OrdColl}.
\end{description}

\subsection{Filters and partitions}
\begin{description}
\spec{filterLT \is a \arrow coll a \arrow coll a \\
      filterLE \is a \arrow coll a \arrow coll a \\
      filterGT \is a \arrow coll a \arrow coll a \\
      filterGE \is a \arrow coll a \arrow coll a}
  Extract the subcollection of elements $<$, $\leq$, $>$, or $\geq$
  the given element.
  Equivalent to the corresponding calls to \cd{filter} (in \cd{Coll}),
  but may be much more efficient.
  \axioms{filterLT x xs \Equiv filter (< x) xs \\
          filterLE x xs \Equiv filter (<= x) xs \\
          filterGT x xs \Equiv filter (> x) xs \\
          filterGE x xs \Equiv filter (>= x) xs}
\spec{partitionLT\_GE \is a \arrow coll a \arrow (coll a, coll a) \\
      partitionLE\_GT \is a \arrow coll a \arrow (coll a, coll a) \\
      partitionLT\_GT \is a \arrow coll a \arrow (coll a, coll a)}
  Split a collection into those elements $<$, $\leq$, or $<$ the
  given element, and those elements $\ge$, $>$, or $>$ the given element.
  \cd{partitionLT\_GE} and \cd{partitionLE\_GT} are equivalent to the 
  corresponding calls to \cd{partition} (in \cd{Coll}), but may be
  much more efficient.  \cd{partitionLT\_GT} cannot be expressed
  as a single call to \cd{partition} because it discards elements
  equal to the given element.
  \axioms{partitionLT\_GE x xs \Equiv partition (< x) xs \\
          partitionLE\_GT x xs \Equiv partition (<= x) xs \\
          partitionLT\_GT x xs \Equiv (filterLT x xs, filterGT x xs)}
\end{description}

\section{\cd{SetX}}
\begin{verbatim}
class CollX c a => SetX c a
\end{verbatim}

\subsection{Set operations}
\begin{description}
\spec{intersect \is coll a \arrow coll a \arrow coll a}
  Computes the intersection of two sets.  It is unspecified which of the
  two elements is kept.
\spec{difference \is coll a \arrow coll a \arrow coll a}
  Computes the difference of two sets (i.e., the set of all elements in the
  first set that are not in the second set).
\spec{subset \is coll a \arrow coll a \arrow Bool \\
      subsetEq \is coll a \arrow coll a \arrow Bool}
  Test whether every element in the first set is also in the second
  set.  \cd{subset} additionally tests whether the second set contains
  at least one element that is not in the first set.
\end{description}


\section{\cd{OrdSetX}}
\begin{verbatim}
class (OrdCollX c a, SetX c a) => OrdSetX c a
\end{verbatim}
%
This class contains no methods.  It exists only as an abbreviation
for the context 
\begin{center}
\cd{(OrdCollX c a, SetX c a)}
\end{center}

\section{\cd{Coll}}
\begin{verbatim}
class CollX c a => Coll c a
\end{verbatim}

\subsection{Observers}
\begin{description}
\spec{lookup \is coll a \arrow a \arrow a \\
      lookupM \is coll a \arrow a \arrow Maybe a \\
      lookupAll \is Sequence seq \Arrow coll a \arrow a \arrow seq a \\
      lookupWithDefault \is a \arrow coll a \arrow a \arrow a}
  Search for an element in the set that is equal to the given element.
  \cd{lookup} signals an error if no such element exists, while
  \cd{lookupWithDefault} returns a default value (provided as
  its first argument).  For bags, it is unspecified which of several
  duplicates is chosen by \cd{lookup}, \cd{lookupM}, or \cd{lookupWithDefault}.
  \cd{lookupAll} returns all the duplicates, but in an unspecified order.

\spec{toSeq \is Sequence seq \Arrow coll a \arrow seq a}
  Return a sequence of all the elements in a collection, in an
  unspecified order.
\end{description}

\subsection{Filters and partitions}
\begin{description}
\spec{filter \is (a \arrow Bool) \arrow coll a \arrow coll a}
  Extract all the elements satisfying the given predicate.
\spec{partition \is (a \arrow Bool) \arrow coll a \arrow (coll a, coll a)}
  Split a collection into those elements satisfying the given predicate,
  and those elements not satisfying the predicate.
\end{description}

\subsection{Folds}
\begin{description}
\spec{fold \is (a \arrow b \arrow b) \arrow b \arrow coll a \arrow b}
  Combine all the elements in a collection into a single value, given
  a combining function and an initial value.  Processes the elements
  in an unspecified order.
\spec{fold1 \is (a \arrow a \arrow a) \arrow coll a \arrow a}
  Combine all the elements in a non-empty collection into a single
  value using the given combining function.  Signals an error if the
  collection is empty.  Processes the elements in an unspecified order.
  An implementation may choose to process the elements linearly or
  in a balanced fashion (like \cd{reduce1} on sequences).
\end{description}


\section{\cd{OrdColl}}
\begin{verbatim}
class (Coll c a, OrdCollX c a) => OrdColl c a
\end{verbatim}

\subsection{Destructors}
\begin{description}
\spec{minView \is coll a \arrow Maybe2 a (coll a) \\
      maxView \is coll a \arrow Maybe2 (coll a) a}
  Separate a collection into its minimum/maximum element and
  the remaining collection.  Return \cd{Nothing2} if the
  collection is empty.  If there is more than one minimum/maximum,
  choose an unspecified one.
\spec{minElem \is coll a \arrow a \\
      maxElem \is coll a \arrow a}
  Return the minimum/maximum element in the collection, or
  signal an error if the collection is empty.  If there is more than
  one minimum/maximum, choose an unspecified one.
\end{description}

\subsection{Observers}
\begin{description}
\spec{toOrdSeq \is Sequence seq \arrow coll a \arrow seq a}
  Convert a collection into a non-decreasing sequence of elements.
  The order in which clusters of equal elements are listed
  is unspecified.  (For sets, the order will always be increasing.)
\end{description}

\subsection{Folds}
\begin{description}
\spec{foldr \is (a \arrow b \arrow b) \arrow b \arrow coll a \arrow b \\
      foldl \is (b \arrow a \arrow b) \arrow b \arrow coll a \arrow b}
  Fold across the elements in non-decreasing order (increasing order
  in the case of sets).  The order in which clusters of equal elements 
  are processed is unspecified.
\spec{foldr1 \is (a \arrow a \arrow a) \arrow coll a \arrow a \\
      foldl1 \is (a \arrow a \arrow a) \arrow coll a \arrow a}
  Fold across the elements in non-decreasing order (increasing order
  in the case of sets), or signal an error if the collection is empty.  
  The order in which clusters of equal elements 
  are processed is unspecified.
\end{description}


\section{\cd{Set}}
\begin{verbatim}
class (Coll c a, SetX c a) => Set c a
\end{verbatim}
%
\emph{Warning: each of the following ``With'' functions is unsafe.}
Each takes a combining function that is used to choose which element
is kept in the case of duplicates.  This combining function must satisfy the
precondition that, given two equal elements, it returns a third element
that is equal to both.  Usually, the combining function just returns
its first or second argument, but it can combine the elements in non-trivial
ways.

The combining function should usually be associative.  If not,
the elements will be combined left-to-right, but with an unspecified
associativity.  For example, it \cd{x~==~y~==~z}, then
\cd{fromList~($\oplus$)~[x,y,z]} equals either
\begin{center}
\cd{single (x $\oplus$ (y $\oplus$ z))}
\hspace{20pt}or\hspace{20pt}
\cd{single ((x $\oplus$ y) $\oplus$ z)}.
\end{center}

\subsection{Constructors}
\begin{description}
\spec{insertWith \is (a \arrow a \arrow a) \arrow a \arrow coll a \arrow coll a \\
      insertSeqWith \is Sequence seq \Arrow (a \arrow a \arrow a) \arrow seq a \arrow coll a \arrow coll a}
  Same as \cd{insert} and \cd{insertSeq}, but with a combining function
  to resolve duplicates.  See the comments about associativity for 
  \cd{insertSeqWith}.
\spec{unionl \is coll a \arrow coll a \arrow coll a \\
      unionr \is coll a \arrow coll a \arrow coll a}
  Same as \cd{union} but keep the element from the left/right collection
  in the case of duplicates.
  \axioms{unionl xs ys \Equiv unionWith ($\lambda$ x y \arrow x) xs ys \\
          unionr xs ys \Equiv unionWith ($\lambda$ x y \arrow y) xs ys}
\spec{unionWith \is (a \arrow a \arrow a) \arrow coll a \arrow coll a \arrow coll a \\
      unionSeqWith \is Sequence seq \Arrow (a \arrow a \arrow a) \arrow seq (coll) a \arrow coll a}
  Same as \cd{union} and \cd{unionSeq}, but with a combining function
  to resolve duplicates.  See the comments about associativity for
  \cd{unionSeqWith}.
\spec{fromSeqWith \is Sequence seq \Arrow (a \arrow a \arrow a) \arrow seq a \arrow coll a}
  Same as \cd{fromSeq}, but with a combining function to resolve duplicates.
  See the comments about associativity.
\end{description}

\subsection{Set operations}
\begin{description}
\spec{intersectWith \is (a \arrow a \arrow a) \arrow coll a \arrow coll a}
  Same as \cd{intersect}, but with a combining function to resolve duplicates.
\end{description}

\section{\cd{OrdSet}}
\begin{verbatim}
class (OrdColl c a, Set c a) => OrdSet c a
\end{verbatim}
%
This class contains no methods.  It exists only as an abbreviation
for the context 
\begin{center}
\cd{(OrdColl c a, Set c a)}
\end{center}

\section{Specialized operations on lists}

For each of the collection methods that involve sequences (e.g.,
\cd{insertSeq}, \cd{toOrdSeq}, \cd{lookupAll}), there is a specialized
version that operates on lists.  The specialized versions are obtained by
replacing the name \cd{Seq} with \cd{List} (e.g., \cd{insertSeq} becomes
\cd{insertList}).  The sole exception to this naming scheme is the
specialized version of \cd{lookupAll}, which is named \cd{lookupList}.

These functions are defined in the \cd{Collections} module and rely
on overloading (as opposed to being accessible directly from each
implementation module).  The types of these functions are
\begin{quote}\ttfamily
\begin{tabular}{l@{$\;$\is}l}
fromList          & CollX c a \Arrow [a] \arrow c a \\
insertList        & CollX c a \Arrow [a] \arrow c a \arrow c a \\
unionList         & CollX c a \Arrow [c a] \arrow c a \\
deleteList        & CollX c a \Arrow [a] \arrow c a \arrow c a \\
unsafeFromOrdList & OrdCollX c a \Arrow [a] \arrow c a \\
toList            & Coll c a \Arrow c a \arrow [a] \\
lookupList        & Coll c a \Arrow c a \arrow a \arrow [a] \\
toOrdList         & OrdColl c a \Arrow c a \arrow [a] \\
fromListWith      & Set c a \Arrow (a \arrow a \arrow a) \arrow [a] \arrow c a \\
insertListWith    & Set c a \Arrow (a \arrow a \arrow a) \arrow [a] \arrow c a \arrow c a \\
unionListWith     & Set c a \Arrow (a \arrow a \arrow a) \arrow [c a] \arrow c a
\end{tabular}
\end{quote}

\section{Utility functions}

The module \cd{CollectionUtils} contains several utility functions.
This module will likely expand in the future.

\begin{description}
\spec{map \is (Coll cin a, CollX cout b) \Arrow (a \arrow b) \arrow (cin a \arrow cout b)}
  Map a function across every element in a collection.  Note that both
  the element type and the collection type may change.
\spec{mapPartial \is (Coll cin a, CollX cout b) \Arrow (a \arrow Maybe b) \arrow (cin a \arrow cout b)}
  Map a partial function across every element in a collection, discarding
  \cd{Nothing} results.  Note that both the element type and the collection
  type may change.
\spec{unsafeMapMonotonic \is (OrdColl cin a, OrdCollX cout b) \Arrow (a \arrow b) \arrow (cin a \arrow cout b)}
  Map a monotonic function across every element in a collection.  Note
  that both the element type and the collection type may change.
  The function \cd{f} must satisfy the precondition that
\begin{center}\ttfamily
x <= y \Implies f x <= f y
\end{center}
  For sets this precondition is strengthened to
\begin{center}\ttfamily
x < y \Implies f x < f y
\end{center}

\spec{unionMap \is (Coll cin a, CollX cout b) \Arrow (a \arrow cout b) \arrow (cin a \arrow cout b)}
  Apply a collection-producing function to every element in a collection,
  and merge the results.  Note that both the element type and the collection
  type may change.  \cd{unionMap} is essentially equivalent to the bind
  operation on monads, except that collections cannot be monads because of the
  use of multi-parameter type classes.
\end{description}

%%%%%%%%%%%%%%%%%%%%%%%%%%%%%%%%%%%%%%%%%%%%%%%%%%%%%%%%%%%%%%%%%%%%%%

\chapter{Associative Collections}

The \emph{associative-collection} abstraction includes finite maps, finite
relations, and priority queues where the priority is separate from element.
Associative collections are defined in Edison as a set of eight classes,
organized in the hierarchy shown in \Figure{assoc-hierarchy}.  Notice that
this hierarchy mirrors the hierarchy for collections, but with the
addition of \cd{Functor} as a superclass of every associative collection.
Like collections, associative collections depend heavily on multi-parameter
type classes.

\begin{figure}
\small
\begin{center}
\begin{picture}(200,200)(0,0)
\put(70,200){\makebox(0,0){\cd{Eq $k$}}}
\put(130,200){\makebox(0,0){\cd{Functor ($m \; k$)}}}
\put(96,168){\line(-1,1){24}}
\put(104,168){\line(1,1){24}}
\put(100,160){\makebox(0,0){\cd{AssocX $m$ $k$}}}
\put(20,118){\line(0,1){24}}
\put(26,118){\line(2,1){68}}
\put(20,150){\makebox(0,0){\cd{Ord $k$}}}
\put(20,110){\makebox(0,0){\cd{OrdAssocX $m$ $k$}}}
\put(174,118){\line(-2,1){68}}
\put(180,110){\makebox(0,0){\cd{FiniteMapX $m$ $k$}}}
\put(106,68){\line(2,1){68}}
\put(94,68){\line(-2,1){68}}
\put(100,60){\makebox(0,0){\cd{OrdFiniteMapX $m$ $k$}}}
\put(100,108){\line(0,0){44}}
\put(100,100){\makebox(0,0){\cd{Assoc $m$ $k$}}}
\put(20,58){\line(0,0){44}}
\put(26,58){\line(2,1){68}}
\put(20,50){\makebox(0,0){\cd{OrdAssoc $m$ $k$}}}
\put(180,58){\line(0,0){44}}
\put(174,58){\line(-2,1){68}}
\put(180,50){\makebox(0,0){\cd{FiniteMap $m$ $k$}}}
\put(106,8){\line(2,1){68}}
\put(94,8){\line(-2,1){68}}
\put(100,0){\makebox(0,0){\cd{OrdFiniteMap $m$ $k$}}}
\end{picture} \\[40pt]

\ttfamily
\begin{tabular}{cccc}
AssocX       & OrdAssocX       & FiniteMapX & OrdFiniteMapX    \\
\hline
empty,insert & minElem         & insertWith & \textit{no methods} \\
union,delete & deleteMin       & unionWith  &                     \\
null,size    & unsafeInsertMin & intersectWith &                  \\
lookup       & foldr,foldl     & difference &                     \\
map,fold     & filterLT        & subset     &                     \\
filter       & $\cdots$        & $\cdots$   &                     \\
$\cdots$     &                 &            &                     \\[20pt]

Assoc        & OrdAssoc        & FiniteMap  & OrdFiniteMap        \\
\hline       
toSeq        & minElemWithKey  & unionWithKey&\textit{no methods} \\
mapWithKey   & foldrWithKey    & intersectWithKey &               \\
foldWithKey  & toOrdSeq        & $\cdots$   &                     \\ 
filterWithKey& $\cdots$        &            &                     \\
$\cdots$     &                 &            &                     \\
\end{tabular}
\end{center}

\normalsize

\caption{The associative-collection class hierarchy, with typical methods for each class.}
\label{assoc-hierarchy}
\end{figure}

The operations on associative collections are similar to the operations
on collections.  The differences arise from having a separate key and
element, rather than just an element.  One significant implication
of this separation is that many of the methods move up in the
hierarchy, because elements are always observable for associative
collections (even though keys may not be).

\begin{figure}
\begin{center}
\large\bfseries Associative-Collection Methods
\end{center}
\hrule
\begin{description}
\item[Constructors:] \nl
\cd{AssocX}: \cd{empty}, \cd{single}, \cd{insert}, \cd{insertSeq}, \cd{union}, \cd{unionSeq}, \cd{fromSeq} \\
\cd{OrdAssocX}: \cd{unsafeInsertMin}, \cd{unsafeInsertMax}, \cd{unsafeFromOrdSeq}, \cd{unsafeAppend} \\
\cd{FiniteMapX}: \cd{insertWith}, \cd{insertWithKey}, \cd{insertSeqWith}, \cd{insertSeqWithKey}, \\
  \hspace*{20pt}
  \cd{unionl}, \cd{unionr}, \cd{unionWith}, \cd{unionSeqWith}, \cd{fromSeqWith}, \cd{fromSeqWithKey} \\
\cd{FiniteMap}: \cd{unionWithKey}, \cd{unionSeqWithKey}
\item[Destructors:] \nl
\cd{OrdAssocX}: \cd{minView}, \cd{minElem}, \cd{maxView}, \cd{maxElem} \\
\cd{OrdAssoc}: \cd{minViewWithKey}, \cd{minElemWithKey}, \cd{maxViewWithKey}, \cd{maxElemWithKey}
\item[Deletions:] \nl
\cd{AssocX}: \cd{delete}, \cd{deleteAll}, \cd{deleteSeq} \\
\cd{OrdAssocX}: \cd{deleteMin}, \cd{deleteMax}
\item[Observers:] \nl
\cd{AssocX}: \cd{null}, \cd{size}, \cd{member}, \cd{count}, \cd{lookup}, \cd{lookupM}, \cd{lookupAll}, \cd{lookupWithDefault}, \cd{elements} \\
\cd{Assoc}: \cd{toSeq}, \cd{keys} \\
\cd{OrdAssoc}: \cd{toOrdSeq}
\item[Modifiers:] \nl
\cd{AssocX}: \cd{adjust}, \cd{adjustAll}
\item[Maps and folds:] \nl
\cd{AssocX}: \cd{map}, \cd{fold}, \cd{fold1} \\
\cd{OrdAssocX}: \cd{foldr}, \cd{foldl}, \cd{foldr1}, \cd{foldl1} \\
\cd{Assoc}: \cd{mapWithKey}, \cd{foldWithKey} \\
\cd{OrdAssoc}: \cd{foldrWithKey}, \cd{foldlWithKey}
\item[Filters and partitions:] \nl
\cd{AssocX}: \cd{filter}, \cd{partition} \\
\cd{OrdAssocX}:
  \cd{filterLT}, \cd{filterLE}, \cd{filterGT}, \cd{filterGE}, \\
  \hspace*{20pt}
  \cd{partitionLT\_GE}, \cd{partitionLE\_GT}, \cd{partitionLT\_GT} \\
\cd{Assoc}: \cd{filterWithKey}, \cd{partitionWithKey}
\item[Set-like operations:] \nl
\cd{FiniteMapX}: \cd{intersectWith}, \cd{difference}, \cd{subset}, \cd{subsetEq} \\
\cd{FiniteMap}: \cd{intersectWithKey}
\end{description}
\hrule
\caption{Summary of methods for the associative-collection classes.}
\label{assoc-methods}
\end{figure}

\Figure{assoc-methods} summarizes all the methods on associative collections.
These methods will be described in more detail in the sections on each subclass
in the hierarchy.

I will frequently abbreviate ``associative collection'' as \AC.  I
will also refer jointly to a key and element as a \emph{binding}.

\section{\cd{AssocX}}
\begin{verbatim}
class Eq k => AssocX m k
\end{verbatim}

\subsection{Constructors}
\begin{description}
\spec{empty \is m k a}
  The empty \AC.

\spec{single \is k \arrow a \arrow m k a}
  Create an \AC with a single binding.

\spec{insert \is k \arrow a \arrow m k a \arrow m k a \\
      insertSeq \is Sequence seq \Arrow seq (k,a) \arrow m k a \arrow m k a}
  Add a binding or a sequence of bindings to a collection.  For finite maps,
  \cd{insert} keeps the new element in the case of duplicate keys, but
  \cd{insertSeq} keeps an unspecified element.  Which key is kept is
  also unspecified for both methods.

  See also \cd{insertWith}, \cd{insertWithKey}, \cd{insertSeqWith},
  and \cd{insertSeqWithKey} in \cd{FiniteMapX}.

\spec{union \is m k a \arrow m k a \arrow m k a \\
      unionSeq \is Sequence seq \Arrow seq (m k a) \arrow m k a}
  Merge two \ACs or a sequence of \ACs.  Which element and which key to
  keep in the case of duplicate keys is unspecified.

  See also \cd{unionWith} and \cd{unionSeqWith} in \cd{FiniteMapX} and
  \cd{unionWithKey} and \cd{unionSeqWithKey} in \cd{FiniteMap}.

\spec{fromSeq \is Sequence seq \Arrow seq (k,a) \arrow m k a}
  Convert a list of bindings to an \AC.  Which element and which key
  to keep in the case of duplicate keys is unspecified.
\end{description}

\subsection{Deletions}
\begin{description}
\spec{delete \is k \arrow m k a \arrow m k a \\
      deleteAll \is k \arrow m k a \arrow m k a}
  Delete one binding or all bindings with the given key, or leave the
  \AC unchanged if it does not contain the key.  For bag-like
  \ACs (i.e., those that allow multiple bindings with the same key), it
  is unspecified which binding will be removed by \cd{delete}.
\spec{deleteSeq \is Sequence seq \Arrow seq k \arrow m k a \arrow m k a}
  Delete a single occurrence of each of the given keys from an \AC,
  ignoring those keys that do not appear in the \AC.  For bag-like
  \ACs, there may be multiple bindings with a given key, in which case
  it is unspecified which is deleted.
\end{description}

\subsection{Observers}
\begin{description}
\spec{null \is m k a \arrow Bool}
  Test whether the \AC is empty.
  \axioms{null m \Equiv (size m == 0)}
\spec{size \is m k a \arrow Int}
  Return the number of bindings in the \AC.
\spec{member \is m k a \arrow k \arrow Bool}
  Test whether the given key is bound in the \AC.
  \axioms{member m k \Equiv (count xs k > 0)}
\spec{count \is m k a \arrow k \arrow Int}
  Return the number of bindings with the given key.
\spec{lookup \is m k a \arrow k \arrow a \\
      lookupM \is m k a \arrow k \arrow Maybe a \\
      lookupAll \is Sequence seq \Arrow m k a \arrow k \arrow seq a \\
      lookupWithDefault \is a \arrow m k a \arrow k \arrow a}
  Find the element associated with the given given.  \cd{lookup}
  signals an error if the key is not bound, while \cd{lookupWithDefault}
  returns a default value (provided as its first argument).
  If there is more than one binding with the given key, it is unspecified
  which element is chosen by \cd{lookup}, \cd{lookupM}, or 
  \cd{lookupWithDefault}.  \cd{lookupAll} returns all elements bound to
  the given key, but in an unspecified order.
\spec{elements \is Sequence seq \Arrow m k a \arrow seq a}
  Return all elements in the given \AC, in an unspecified order.
\end{description}

\subsection{Modifiers}
\begin{description}
\spec{adjust \is (a \arrow a) \arrow k \arrow m k a \arrow m k a \\
      adjustAll \is (a \arrow a) \arrow k \arrow m k a \arrow m k a}
  Change a single binding or all bindings for the given key by applying
  a function to its element(s).  If more than one binding has the
  same key, it is unspecified which is modified by \cd{adjust}.
\end{description}

\subsection{Maps and folds}
\begin{description}
\spec{map \is (a \arrow b) \arrow m k a \arrow m k b}
  Apply a function to the elements of every binding in an \AC.
  This method appears both in \cd{AssocX} and in \cd{Functor}.
\spec{fold \is (a \arrow b \arrow b) \arrow b \arrow m k a \arrow b}
  Combine all the elements in an \AC, given a combining function and
  an initial value.  The elements are processed in an unspecified order.
  Note that \cd{fold} ignores the keys.
\spec{fold1 \is (a \arrow a \arrow a) \arrow m k a \arrow a}
  Combine all the elements in a non-empty \AC using the given combining
  function.  Signals an error if the \AC is empty.  The elements
  are processed in an unspecified order.  An implementation may choose to
  process the elements linearly or in a balanced fashion (like \cd{reduce1}
  on sequences).  Note that \cd{fold1} ignores the keys.
\end{description}

\subsection{Filters and partitions}
\begin{description}
\spec{filter \is (a \arrow Bool) \arrow m k a \arrow m k a}
  Extract all the bindings whose elements satisfy the given predicate.
\spec{partition \is (a \arrow Bool) \arrow m k a \arrow (m k a, m k a)}
  Split an \AC into those bindings whose elements satisfy the given predicate,
  and those bindings whose elements do not satisfy the predicate.
\end{description}


\section{\cd{OrdAssocX}}
\begin{verbatim}
class (AssocX m k, Ord k) => OrdAssocX m k
\end{verbatim}

\subsection{Constructors}
\begin{description}
\spec{unsafeInsertMin \is k \arrow a \arrow m k a \arrow m k a \\
      unsafeInsertMax \is m k a \arrow k \arrow a \arrow m k a}
  Insert a binding into an \AC with the precondition that the given key
  is $\leq$ or $\geq$ any exisiting keys.  For finite maps, this precondition
  is strengthened to $<$ or $>$.
\spec{unsafeFromOrdSeq \is Sequence seq \arrow seq (k,a) \arrow m k a}
  Convert a sequence of bindings into an \AC with the precondition that
  the sequence is sorted into non-decreasing order by key.  For finite
  maps, this precondition is strengthened to increasing order.
\spec{unsafeAppend \is m k a \arrow m k a \arrow m k a}
  Merge two \ACs with the precondition that every key in the first
  \AC is $\leq$ every key in the second \AC.  For finite maps, this precondition
  is strengthened to $<$.
\end{description}

\subsection{Destructors}
\begin{description}
\spec{minView \is m k a \arrow Maybe2 a (m k a) \\
      maxView \is m k a \arrow Maybe2 (m k a) a}
  Remove the binding with the minimum or maximum key, and return its element 
  together with the remaining \AC.  Return \cd{Nothing2} if the
  \AC is empty.  Which binding is removed if there is more than one minimum
  or maximum is unspecified.

  See also \cd{minViewWithKey} and \cd{maxViewWithKey} in \cd{OrdAssoc}.
  
\spec{minElem \is m k a \arrow a \\
      maxElem \is m k a \arrow a}
  Return the element associated with the minimum or maximum key, or
  signal an error if the \AC is empty.  Which element is chosen if
  there is more than one minimum or maximum is unspecified.

  See also \cd{minElemWithKey} and \cd{maxElemWithKey} in \cd{OrdAssoc}.
\end{description}

\subsection{Deletions}
\begin{description}
\spec{deleteMin \is m k a \arrow m k a \\
      deleteMax \is m k a \arrow m k a}
  Delete the binding with the minimum or maximum key, or return \cd{empty}
  if the \AC is already empty.  Which binding is chosen if there is
  more than one minimum or maximum is unspecified.
\end{description}

\subsection{Maps and Folds}
\begin{description}
\spec{foldr \is (a \arrow b \arrow b) \arrow b \arrow m k a \arrow b \\
      foldl \is (b \arrow a \arrow b) \arrow b \arrow m k a \arrow b}
  Fold across the elements in non-decreasing order by key.
\spec{foldr1 \is (a \arrow a \arrow a) \arrow m k a \arrow a \\
      foldl1 \is (a \arrow a \arrow a) \arrow m k a \arrow a}
  Fold across the elements in non-decreasing order by key, or signal
  an error if the \AC is empty.
\end{description}

\subsection{Filters and partitions}
\begin{description}
\spec{filterLT \\
      filterLE \\
      filterGT \\
      filterGE}
\spec{partitionLT\_GE \\
      partitionLE\_GT \\
      partitionLT\_GT}
\end{description}


\section{\cd{FiniteMapX}}
\begin{verbatim}
class AssocX m k => FiniteMapX m k
\end{verbatim}

\subsection{Constructors}
\begin{description}
\spec{insertWith \\
      insertSeqWith}
\spec{insertWithKey \\
      insertSeqWithKey}
\spec{unionl \\
      unionr}
\spec{unionWith \\
      unionSeqWith}
\spec{fromSeqWith \\
      fromSeqWithKey}
\end{description}

\subsection{Set-like operations}
\begin{description}
\spec{intersectWith}
\spec{difference}
\spec{subset}
\spec{subsetEq}
\end{description}

\section{\cd{OrdFiniteMapX}}
\begin{verbatim}
class (OrdAssocX m k, FiniteMapX m k) => OrdFiniteMapX m k
\end{verbatim}
%
This class contains no methods.  It exists only as an abbreviation
for the context 
\begin{center}
\cd{(OrdAssocX m k, FiniteMapX m k)}
\end{center}

\section{\cd{Assoc}}
\begin{verbatim}
class AssocX m k => Assoc m k
\end{verbatim}

\subsection{Observers}
\begin{description}
\spec{toSeq}
\spec{keys}
\end{description}

\subsection{Maps and folds}
\begin{description}
\spec{mapWithKey}
\spec{foldWithKey}
\end{description}

\subsection{Filters and partitions}
\begin{description}
\spec{filterWithKey}
\spec{partitionWithKey}
\end{description}

\section{\cd{OrdAssoc}}
\begin{verbatim}
class (Assoc m k, OrdAssocX m k) => OrdAssoc m k
\end{verbatim}

\subsection{Destructors}
\begin{description}
\spec{minViewWithKey}
\spec{minElemWithKey}
\spec{maxViewWithKey}
\spec{maxElemWithKey}
\end{description}

\subsection{Observers}
\begin{description}
\spec{toOrdSeq}
\end{description}

\subsection{Maps and folds}
\begin{description}
\spec{foldrWithKey \\
      foldlWithKey}
\end{description}

\section{\cd{FiniteMap}}
\begin{verbatim}
class (Assoc m k, FiniteMapX m k) => FiniteMap m k
\end{verbatim}

\subsection{Constructors}
\begin{description}
\spec{unionWithKey \\
      unionSeqWithKey}
\end{description}

\subsection{Set-like operations}
\begin{description}
\spec{intersectWithKey}
\end{description}

\section{\cd{OrdFiniteMap}}
\begin{verbatim}
class (OrdAssoc m k, FiniteMap m k) => OrdFiniteMap m k
\end{verbatim}
%
This class contains no methods.  It exists only as an abbreviation
for the context 
\begin{center}
\cd{(OrdAssoc m k, FiniteMap m k)}
\end{center}

\end{document}





